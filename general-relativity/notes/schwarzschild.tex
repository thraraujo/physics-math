\documentclass[11pt]{article}

    \usepackage[breakable]{tcolorbox}
    \usepackage{parskip} % Stop auto-indenting (to mimic markdown behaviour)
    

    % Basic figure setup, for now with no caption control since it's done
    % automatically by Pandoc (which extracts ![](path) syntax from Markdown).
    \usepackage{graphicx}
    % Maintain compatibility with old templates. Remove in nbconvert 6.0
    \let\Oldincludegraphics\includegraphics
    % Ensure that by default, figures have no caption (until we provide a
    % proper Figure object with a Caption API and a way to capture that
    % in the conversion process - todo).
    \usepackage{caption}
    \DeclareCaptionFormat{nocaption}{}
    \captionsetup{format=nocaption,aboveskip=0pt,belowskip=0pt}

    \usepackage{float}
    \floatplacement{figure}{H} % forces figures to be placed at the correct location
    \usepackage{xcolor} % Allow colors to be defined
    \usepackage{enumerate} % Needed for markdown enumerations to work
    \usepackage{geometry} % Used to adjust the document margins
    \usepackage{amsmath} % Equations
    \usepackage{amssymb} % Equations
    \usepackage{textcomp} % defines textquotesingle
    % Hack from http://tex.stackexchange.com/a/47451/13684:
    \AtBeginDocument{%
        \def\PYZsq{\textquotesingle}% Upright quotes in Pygmentized code
    }
    \usepackage{upquote} % Upright quotes for verbatim code
    \usepackage{eurosym} % defines \euro

    \usepackage{iftex}
    \ifPDFTeX
        \usepackage[T1]{fontenc}
        \IfFileExists{alphabeta.sty}{
              \usepackage{alphabeta}
          }{
              \usepackage[mathletters]{ucs}
              \usepackage[utf8x]{inputenc}
          }
    \else
        \usepackage{fontspec}
        \usepackage{unicode-math}
    \fi

    \usepackage{fancyvrb} % verbatim replacement that allows latex
    \usepackage{grffile} % extends the file name processing of package graphics
                         % to support a larger range
    \makeatletter % fix for old versions of grffile with XeLaTeX
    \@ifpackagelater{grffile}{2019/11/01}
    {
      % Do nothing on new versions
    }
    {
      \def\Gread@@xetex#1{%
        \IfFileExists{"\Gin@base".bb}%
        {\Gread@eps{\Gin@base.bb}}%
        {\Gread@@xetex@aux#1}%
      }
    }
    \makeatother
    \usepackage[Export]{adjustbox} % Used to constrain images to a maximum size
    \adjustboxset{max size={0.9\linewidth}{0.9\paperheight}}

    % The hyperref package gives us a pdf with properly built
    % internal navigation ('pdf bookmarks' for the table of contents,
    % internal cross-reference links, web links for URLs, etc.)
    \usepackage{hyperref}
    % The default LaTeX title has an obnoxious amount of whitespace. By default,
    % titling removes some of it. It also provides customization options.
    \usepackage{titling}
    \usepackage{longtable} % longtable support required by pandoc >1.10
    \usepackage{booktabs}  % table support for pandoc > 1.12.2
    \usepackage{array}     % table support for pandoc >= 2.11.3
    \usepackage{calc}      % table minipage width calculation for pandoc >= 2.11.1
    \usepackage[inline]{enumitem} % IRkernel/repr support (it uses the enumerate* environment)
    \usepackage[normalem]{ulem} % ulem is needed to support strikethroughs (\sout)
                                % normalem makes italics be italics, not underlines
    \usepackage{mathrsfs}
    

    
    % Colors for the hyperref package
    \definecolor{urlcolor}{rgb}{0,.145,.698}
    \definecolor{linkcolor}{rgb}{.71,0.21,0.01}
    \definecolor{citecolor}{rgb}{.12,.54,.11}

    % ANSI colors
    \definecolor{ansi-black}{HTML}{3E424D}
    \definecolor{ansi-black-intense}{HTML}{282C36}
    \definecolor{ansi-red}{HTML}{E75C58}
    \definecolor{ansi-red-intense}{HTML}{B22B31}
    \definecolor{ansi-green}{HTML}{00A250}
    \definecolor{ansi-green-intense}{HTML}{007427}
    \definecolor{ansi-yellow}{HTML}{DDB62B}
    \definecolor{ansi-yellow-intense}{HTML}{B27D12}
    \definecolor{ansi-blue}{HTML}{208FFB}
    \definecolor{ansi-blue-intense}{HTML}{0065CA}
    \definecolor{ansi-magenta}{HTML}{D160C4}
    \definecolor{ansi-magenta-intense}{HTML}{A03196}
    \definecolor{ansi-cyan}{HTML}{60C6C8}
    \definecolor{ansi-cyan-intense}{HTML}{258F8F}
    \definecolor{ansi-white}{HTML}{C5C1B4}
    \definecolor{ansi-white-intense}{HTML}{A1A6B2}
    \definecolor{ansi-default-inverse-fg}{HTML}{FFFFFF}
    \definecolor{ansi-default-inverse-bg}{HTML}{000000}

    % common color for the border for error outputs.
    \definecolor{outerrorbackground}{HTML}{FFDFDF}

    % commands and environments needed by pandoc snippets
    % extracted from the output of `pandoc -s`
    \providecommand{\tightlist}{%
      \setlength{\itemsep}{0pt}\setlength{\parskip}{0pt}}
    \DefineVerbatimEnvironment{Highlighting}{Verbatim}{commandchars=\\\{\}}
    % Add ',fontsize=\small' for more characters per line
    \newenvironment{Shaded}{}{}
    \newcommand{\KeywordTok}[1]{\textcolor[rgb]{0.00,0.44,0.13}{\textbf{{#1}}}}
    \newcommand{\DataTypeTok}[1]{\textcolor[rgb]{0.56,0.13,0.00}{{#1}}}
    \newcommand{\DecValTok}[1]{\textcolor[rgb]{0.25,0.63,0.44}{{#1}}}
    \newcommand{\BaseNTok}[1]{\textcolor[rgb]{0.25,0.63,0.44}{{#1}}}
    \newcommand{\FloatTok}[1]{\textcolor[rgb]{0.25,0.63,0.44}{{#1}}}
    \newcommand{\CharTok}[1]{\textcolor[rgb]{0.25,0.44,0.63}{{#1}}}
    \newcommand{\StringTok}[1]{\textcolor[rgb]{0.25,0.44,0.63}{{#1}}}
    \newcommand{\CommentTok}[1]{\textcolor[rgb]{0.38,0.63,0.69}{\textit{{#1}}}}
    \newcommand{\OtherTok}[1]{\textcolor[rgb]{0.00,0.44,0.13}{{#1}}}
    \newcommand{\AlertTok}[1]{\textcolor[rgb]{1.00,0.00,0.00}{\textbf{{#1}}}}
    \newcommand{\FunctionTok}[1]{\textcolor[rgb]{0.02,0.16,0.49}{{#1}}}
    \newcommand{\RegionMarkerTok}[1]{{#1}}
    \newcommand{\ErrorTok}[1]{\textcolor[rgb]{1.00,0.00,0.00}{\textbf{{#1}}}}
    \newcommand{\NormalTok}[1]{{#1}}

    % Additional commands for more recent versions of Pandoc
    \newcommand{\ConstantTok}[1]{\textcolor[rgb]{0.53,0.00,0.00}{{#1}}}
    \newcommand{\SpecialCharTok}[1]{\textcolor[rgb]{0.25,0.44,0.63}{{#1}}}
    \newcommand{\VerbatimStringTok}[1]{\textcolor[rgb]{0.25,0.44,0.63}{{#1}}}
    \newcommand{\SpecialStringTok}[1]{\textcolor[rgb]{0.73,0.40,0.53}{{#1}}}
    \newcommand{\ImportTok}[1]{{#1}}
    \newcommand{\DocumentationTok}[1]{\textcolor[rgb]{0.73,0.13,0.13}{\textit{{#1}}}}
    \newcommand{\AnnotationTok}[1]{\textcolor[rgb]{0.38,0.63,0.69}{\textbf{\textit{{#1}}}}}
    \newcommand{\CommentVarTok}[1]{\textcolor[rgb]{0.38,0.63,0.69}{\textbf{\textit{{#1}}}}}
    \newcommand{\VariableTok}[1]{\textcolor[rgb]{0.10,0.09,0.49}{{#1}}}
    \newcommand{\ControlFlowTok}[1]{\textcolor[rgb]{0.00,0.44,0.13}{\textbf{{#1}}}}
    \newcommand{\OperatorTok}[1]{\textcolor[rgb]{0.40,0.40,0.40}{{#1}}}
    \newcommand{\BuiltInTok}[1]{{#1}}
    \newcommand{\ExtensionTok}[1]{{#1}}
    \newcommand{\PreprocessorTok}[1]{\textcolor[rgb]{0.74,0.48,0.00}{{#1}}}
    \newcommand{\AttributeTok}[1]{\textcolor[rgb]{0.49,0.56,0.16}{{#1}}}
    \newcommand{\InformationTok}[1]{\textcolor[rgb]{0.38,0.63,0.69}{\textbf{\textit{{#1}}}}}
    \newcommand{\WarningTok}[1]{\textcolor[rgb]{0.38,0.63,0.69}{\textbf{\textit{{#1}}}}}


    % Define a nice break command that doesn't care if a line doesn't already
    % exist.
    \def\br{\hspace*{\fill} \\* }
    % Math Jax compatibility definitions
    \def\gt{>}
    \def\lt{<}
    \let\Oldtex\TeX
    \let\Oldlatex\LaTeX
    \renewcommand{\TeX}{\textrm{\Oldtex}}
    \renewcommand{\LaTeX}{\textrm{\Oldlatex}}
    % Document parameters
    % Document title
    \title{schwarzschild}
    
    
    
    
    
% Pygments definitions
\makeatletter
\def\PY@reset{\let\PY@it=\relax \let\PY@bf=\relax%
    \let\PY@ul=\relax \let\PY@tc=\relax%
    \let\PY@bc=\relax \let\PY@ff=\relax}
\def\PY@tok#1{\csname PY@tok@#1\endcsname}
\def\PY@toks#1+{\ifx\relax#1\empty\else%
    \PY@tok{#1}\expandafter\PY@toks\fi}
\def\PY@do#1{\PY@bc{\PY@tc{\PY@ul{%
    \PY@it{\PY@bf{\PY@ff{#1}}}}}}}
\def\PY#1#2{\PY@reset\PY@toks#1+\relax+\PY@do{#2}}

\@namedef{PY@tok@w}{\def\PY@tc##1{\textcolor[rgb]{0.73,0.73,0.73}{##1}}}
\@namedef{PY@tok@c}{\let\PY@it=\textit\def\PY@tc##1{\textcolor[rgb]{0.24,0.48,0.48}{##1}}}
\@namedef{PY@tok@cp}{\def\PY@tc##1{\textcolor[rgb]{0.61,0.40,0.00}{##1}}}
\@namedef{PY@tok@k}{\let\PY@bf=\textbf\def\PY@tc##1{\textcolor[rgb]{0.00,0.50,0.00}{##1}}}
\@namedef{PY@tok@kp}{\def\PY@tc##1{\textcolor[rgb]{0.00,0.50,0.00}{##1}}}
\@namedef{PY@tok@kt}{\def\PY@tc##1{\textcolor[rgb]{0.69,0.00,0.25}{##1}}}
\@namedef{PY@tok@o}{\def\PY@tc##1{\textcolor[rgb]{0.40,0.40,0.40}{##1}}}
\@namedef{PY@tok@ow}{\let\PY@bf=\textbf\def\PY@tc##1{\textcolor[rgb]{0.67,0.13,1.00}{##1}}}
\@namedef{PY@tok@nb}{\def\PY@tc##1{\textcolor[rgb]{0.00,0.50,0.00}{##1}}}
\@namedef{PY@tok@nf}{\def\PY@tc##1{\textcolor[rgb]{0.00,0.00,1.00}{##1}}}
\@namedef{PY@tok@nc}{\let\PY@bf=\textbf\def\PY@tc##1{\textcolor[rgb]{0.00,0.00,1.00}{##1}}}
\@namedef{PY@tok@nn}{\let\PY@bf=\textbf\def\PY@tc##1{\textcolor[rgb]{0.00,0.00,1.00}{##1}}}
\@namedef{PY@tok@ne}{\let\PY@bf=\textbf\def\PY@tc##1{\textcolor[rgb]{0.80,0.25,0.22}{##1}}}
\@namedef{PY@tok@nv}{\def\PY@tc##1{\textcolor[rgb]{0.10,0.09,0.49}{##1}}}
\@namedef{PY@tok@no}{\def\PY@tc##1{\textcolor[rgb]{0.53,0.00,0.00}{##1}}}
\@namedef{PY@tok@nl}{\def\PY@tc##1{\textcolor[rgb]{0.46,0.46,0.00}{##1}}}
\@namedef{PY@tok@ni}{\let\PY@bf=\textbf\def\PY@tc##1{\textcolor[rgb]{0.44,0.44,0.44}{##1}}}
\@namedef{PY@tok@na}{\def\PY@tc##1{\textcolor[rgb]{0.41,0.47,0.13}{##1}}}
\@namedef{PY@tok@nt}{\let\PY@bf=\textbf\def\PY@tc##1{\textcolor[rgb]{0.00,0.50,0.00}{##1}}}
\@namedef{PY@tok@nd}{\def\PY@tc##1{\textcolor[rgb]{0.67,0.13,1.00}{##1}}}
\@namedef{PY@tok@s}{\def\PY@tc##1{\textcolor[rgb]{0.73,0.13,0.13}{##1}}}
\@namedef{PY@tok@sd}{\let\PY@it=\textit\def\PY@tc##1{\textcolor[rgb]{0.73,0.13,0.13}{##1}}}
\@namedef{PY@tok@si}{\let\PY@bf=\textbf\def\PY@tc##1{\textcolor[rgb]{0.64,0.35,0.47}{##1}}}
\@namedef{PY@tok@se}{\let\PY@bf=\textbf\def\PY@tc##1{\textcolor[rgb]{0.67,0.36,0.12}{##1}}}
\@namedef{PY@tok@sr}{\def\PY@tc##1{\textcolor[rgb]{0.64,0.35,0.47}{##1}}}
\@namedef{PY@tok@ss}{\def\PY@tc##1{\textcolor[rgb]{0.10,0.09,0.49}{##1}}}
\@namedef{PY@tok@sx}{\def\PY@tc##1{\textcolor[rgb]{0.00,0.50,0.00}{##1}}}
\@namedef{PY@tok@m}{\def\PY@tc##1{\textcolor[rgb]{0.40,0.40,0.40}{##1}}}
\@namedef{PY@tok@gh}{\let\PY@bf=\textbf\def\PY@tc##1{\textcolor[rgb]{0.00,0.00,0.50}{##1}}}
\@namedef{PY@tok@gu}{\let\PY@bf=\textbf\def\PY@tc##1{\textcolor[rgb]{0.50,0.00,0.50}{##1}}}
\@namedef{PY@tok@gd}{\def\PY@tc##1{\textcolor[rgb]{0.63,0.00,0.00}{##1}}}
\@namedef{PY@tok@gi}{\def\PY@tc##1{\textcolor[rgb]{0.00,0.52,0.00}{##1}}}
\@namedef{PY@tok@gr}{\def\PY@tc##1{\textcolor[rgb]{0.89,0.00,0.00}{##1}}}
\@namedef{PY@tok@ge}{\let\PY@it=\textit}
\@namedef{PY@tok@gs}{\let\PY@bf=\textbf}
\@namedef{PY@tok@gp}{\let\PY@bf=\textbf\def\PY@tc##1{\textcolor[rgb]{0.00,0.00,0.50}{##1}}}
\@namedef{PY@tok@go}{\def\PY@tc##1{\textcolor[rgb]{0.44,0.44,0.44}{##1}}}
\@namedef{PY@tok@gt}{\def\PY@tc##1{\textcolor[rgb]{0.00,0.27,0.87}{##1}}}
\@namedef{PY@tok@err}{\def\PY@bc##1{{\setlength{\fboxsep}{\string -\fboxrule}\fcolorbox[rgb]{1.00,0.00,0.00}{1,1,1}{\strut ##1}}}}
\@namedef{PY@tok@kc}{\let\PY@bf=\textbf\def\PY@tc##1{\textcolor[rgb]{0.00,0.50,0.00}{##1}}}
\@namedef{PY@tok@kd}{\let\PY@bf=\textbf\def\PY@tc##1{\textcolor[rgb]{0.00,0.50,0.00}{##1}}}
\@namedef{PY@tok@kn}{\let\PY@bf=\textbf\def\PY@tc##1{\textcolor[rgb]{0.00,0.50,0.00}{##1}}}
\@namedef{PY@tok@kr}{\let\PY@bf=\textbf\def\PY@tc##1{\textcolor[rgb]{0.00,0.50,0.00}{##1}}}
\@namedef{PY@tok@bp}{\def\PY@tc##1{\textcolor[rgb]{0.00,0.50,0.00}{##1}}}
\@namedef{PY@tok@fm}{\def\PY@tc##1{\textcolor[rgb]{0.00,0.00,1.00}{##1}}}
\@namedef{PY@tok@vc}{\def\PY@tc##1{\textcolor[rgb]{0.10,0.09,0.49}{##1}}}
\@namedef{PY@tok@vg}{\def\PY@tc##1{\textcolor[rgb]{0.10,0.09,0.49}{##1}}}
\@namedef{PY@tok@vi}{\def\PY@tc##1{\textcolor[rgb]{0.10,0.09,0.49}{##1}}}
\@namedef{PY@tok@vm}{\def\PY@tc##1{\textcolor[rgb]{0.10,0.09,0.49}{##1}}}
\@namedef{PY@tok@sa}{\def\PY@tc##1{\textcolor[rgb]{0.73,0.13,0.13}{##1}}}
\@namedef{PY@tok@sb}{\def\PY@tc##1{\textcolor[rgb]{0.73,0.13,0.13}{##1}}}
\@namedef{PY@tok@sc}{\def\PY@tc##1{\textcolor[rgb]{0.73,0.13,0.13}{##1}}}
\@namedef{PY@tok@dl}{\def\PY@tc##1{\textcolor[rgb]{0.73,0.13,0.13}{##1}}}
\@namedef{PY@tok@s2}{\def\PY@tc##1{\textcolor[rgb]{0.73,0.13,0.13}{##1}}}
\@namedef{PY@tok@sh}{\def\PY@tc##1{\textcolor[rgb]{0.73,0.13,0.13}{##1}}}
\@namedef{PY@tok@s1}{\def\PY@tc##1{\textcolor[rgb]{0.73,0.13,0.13}{##1}}}
\@namedef{PY@tok@mb}{\def\PY@tc##1{\textcolor[rgb]{0.40,0.40,0.40}{##1}}}
\@namedef{PY@tok@mf}{\def\PY@tc##1{\textcolor[rgb]{0.40,0.40,0.40}{##1}}}
\@namedef{PY@tok@mh}{\def\PY@tc##1{\textcolor[rgb]{0.40,0.40,0.40}{##1}}}
\@namedef{PY@tok@mi}{\def\PY@tc##1{\textcolor[rgb]{0.40,0.40,0.40}{##1}}}
\@namedef{PY@tok@il}{\def\PY@tc##1{\textcolor[rgb]{0.40,0.40,0.40}{##1}}}
\@namedef{PY@tok@mo}{\def\PY@tc##1{\textcolor[rgb]{0.40,0.40,0.40}{##1}}}
\@namedef{PY@tok@ch}{\let\PY@it=\textit\def\PY@tc##1{\textcolor[rgb]{0.24,0.48,0.48}{##1}}}
\@namedef{PY@tok@cm}{\let\PY@it=\textit\def\PY@tc##1{\textcolor[rgb]{0.24,0.48,0.48}{##1}}}
\@namedef{PY@tok@cpf}{\let\PY@it=\textit\def\PY@tc##1{\textcolor[rgb]{0.24,0.48,0.48}{##1}}}
\@namedef{PY@tok@c1}{\let\PY@it=\textit\def\PY@tc##1{\textcolor[rgb]{0.24,0.48,0.48}{##1}}}
\@namedef{PY@tok@cs}{\let\PY@it=\textit\def\PY@tc##1{\textcolor[rgb]{0.24,0.48,0.48}{##1}}}

\def\PYZbs{\char`\\}
\def\PYZus{\char`\_}
\def\PYZob{\char`\{}
\def\PYZcb{\char`\}}
\def\PYZca{\char`\^}
\def\PYZam{\char`\&}
\def\PYZlt{\char`\<}
\def\PYZgt{\char`\>}
\def\PYZsh{\char`\#}
\def\PYZpc{\char`\%}
\def\PYZdl{\char`\$}
\def\PYZhy{\char`\-}
\def\PYZsq{\char`\'}
\def\PYZdq{\char`\"}
\def\PYZti{\char`\~}
% for compatibility with earlier versions
\def\PYZat{@}
\def\PYZlb{[}
\def\PYZrb{]}
\makeatother


    % For linebreaks inside Verbatim environment from package fancyvrb.
    \makeatletter
        \newbox\Wrappedcontinuationbox
        \newbox\Wrappedvisiblespacebox
        \newcommand*\Wrappedvisiblespace {\textcolor{red}{\textvisiblespace}}
        \newcommand*\Wrappedcontinuationsymbol {\textcolor{red}{\llap{\tiny$\m@th\hookrightarrow$}}}
        \newcommand*\Wrappedcontinuationindent {3ex }
        \newcommand*\Wrappedafterbreak {\kern\Wrappedcontinuationindent\copy\Wrappedcontinuationbox}
        % Take advantage of the already applied Pygments mark-up to insert
        % potential linebreaks for TeX processing.
        %        {, <, #, %, $, ' and ": go to next line.
        %        _, }, ^, &, >, - and ~: stay at end of broken line.
        % Use of \textquotesingle for straight quote.
        \newcommand*\Wrappedbreaksatspecials {%
            \def\PYGZus{\discretionary{\char`\_}{\Wrappedafterbreak}{\char`\_}}%
            \def\PYGZob{\discretionary{}{\Wrappedafterbreak\char`\{}{\char`\{}}%
            \def\PYGZcb{\discretionary{\char`\}}{\Wrappedafterbreak}{\char`\}}}%
            \def\PYGZca{\discretionary{\char`\^}{\Wrappedafterbreak}{\char`\^}}%
            \def\PYGZam{\discretionary{\char`\&}{\Wrappedafterbreak}{\char`\&}}%
            \def\PYGZlt{\discretionary{}{\Wrappedafterbreak\char`\<}{\char`\<}}%
            \def\PYGZgt{\discretionary{\char`\>}{\Wrappedafterbreak}{\char`\>}}%
            \def\PYGZsh{\discretionary{}{\Wrappedafterbreak\char`\#}{\char`\#}}%
            \def\PYGZpc{\discretionary{}{\Wrappedafterbreak\char`\%}{\char`\%}}%
            \def\PYGZdl{\discretionary{}{\Wrappedafterbreak\char`\$}{\char`\$}}%
            \def\PYGZhy{\discretionary{\char`\-}{\Wrappedafterbreak}{\char`\-}}%
            \def\PYGZsq{\discretionary{}{\Wrappedafterbreak\textquotesingle}{\textquotesingle}}%
            \def\PYGZdq{\discretionary{}{\Wrappedafterbreak\char`\"}{\char`\"}}%
            \def\PYGZti{\discretionary{\char`\~}{\Wrappedafterbreak}{\char`\~}}%
        }
        % Some characters . , ; ? ! / are not pygmentized.
        % This macro makes them "active" and they will insert potential linebreaks
        \newcommand*\Wrappedbreaksatpunct {%
            \lccode`\~`\.\lowercase{\def~}{\discretionary{\hbox{\char`\.}}{\Wrappedafterbreak}{\hbox{\char`\.}}}%
            \lccode`\~`\,\lowercase{\def~}{\discretionary{\hbox{\char`\,}}{\Wrappedafterbreak}{\hbox{\char`\,}}}%
            \lccode`\~`\;\lowercase{\def~}{\discretionary{\hbox{\char`\;}}{\Wrappedafterbreak}{\hbox{\char`\;}}}%
            \lccode`\~`\:\lowercase{\def~}{\discretionary{\hbox{\char`\:}}{\Wrappedafterbreak}{\hbox{\char`\:}}}%
            \lccode`\~`\?\lowercase{\def~}{\discretionary{\hbox{\char`\?}}{\Wrappedafterbreak}{\hbox{\char`\?}}}%
            \lccode`\~`\!\lowercase{\def~}{\discretionary{\hbox{\char`\!}}{\Wrappedafterbreak}{\hbox{\char`\!}}}%
            \lccode`\~`\/\lowercase{\def~}{\discretionary{\hbox{\char`\/}}{\Wrappedafterbreak}{\hbox{\char`\/}}}%
            \catcode`\.\active
            \catcode`\,\active
            \catcode`\;\active
            \catcode`\:\active
            \catcode`\?\active
            \catcode`\!\active
            \catcode`\/\active
            \lccode`\~`\~
        }
    \makeatother

    \let\OriginalVerbatim=\Verbatim
    \makeatletter
    \renewcommand{\Verbatim}[1][1]{%
        %\parskip\z@skip
        \sbox\Wrappedcontinuationbox {\Wrappedcontinuationsymbol}%
        \sbox\Wrappedvisiblespacebox {\FV@SetupFont\Wrappedvisiblespace}%
        \def\FancyVerbFormatLine ##1{\hsize\linewidth
            \vtop{\raggedright\hyphenpenalty\z@\exhyphenpenalty\z@
                \doublehyphendemerits\z@\finalhyphendemerits\z@
                \strut ##1\strut}%
        }%
        % If the linebreak is at a space, the latter will be displayed as visible
        % space at end of first line, and a continuation symbol starts next line.
        % Stretch/shrink are however usually zero for typewriter font.
        \def\FV@Space {%
            \nobreak\hskip\z@ plus\fontdimen3\font minus\fontdimen4\font
            \discretionary{\copy\Wrappedvisiblespacebox}{\Wrappedafterbreak}
            {\kern\fontdimen2\font}%
        }%

        % Allow breaks at special characters using \PYG... macros.
        \Wrappedbreaksatspecials
        % Breaks at punctuation characters . , ; ? ! and / need catcode=\active
        \OriginalVerbatim[#1,codes*=\Wrappedbreaksatpunct]%
    }
    \makeatother

    % Exact colors from NB
    \definecolor{incolor}{HTML}{303F9F}
    \definecolor{outcolor}{HTML}{D84315}
    \definecolor{cellborder}{HTML}{CFCFCF}
    \definecolor{cellbackground}{HTML}{F7F7F7}

    % prompt
    \makeatletter
    \newcommand{\boxspacing}{\kern\kvtcb@left@rule\kern\kvtcb@boxsep}
    \makeatother
    \newcommand{\prompt}[4]{
        {\ttfamily\llap{{\color{#2}[#3]:\hspace{3pt}#4}}\vspace{-\baselineskip}}
    }
    

    
    % Prevent overflowing lines due to hard-to-break entities
    \sloppy
    % Setup hyperref package
    \hypersetup{
      breaklinks=true,  % so long urls are correctly broken across lines
      colorlinks=true,
      urlcolor=urlcolor,
      linkcolor=linkcolor,
      citecolor=citecolor,
      }
    % Slightly bigger margins than the latex defaults
    
    \geometry{verbose,tmargin=1in,bmargin=1in,lmargin=1in,rmargin=1in}
    
    

\begin{document}
    
    \maketitle
    
    

    
    \hypertarget{introduuxe7uxe3o-e-motivauxe7uxe3o}{%
\section{Introdução e
Motivação}\label{introduuxe7uxe3o-e-motivauxe7uxe3o}}

    O espaço-tempo de Schwarzschild é a solução exata (e não-trivial) mais
conhecida das equações de Einstein. Foi a primeira solução exata a ser
descoberta poucos meses após a publicação dos artigos originais de
Einstein. E felizmente, essa é a solução mais importante (não é exagero)
da equação de Einstein.

Esse espaço-tempo descreve o campo gravitacional \textbf{exterior}
gerado por um corpo perfeitamente esférico, sem carga elétrica e que não
está girando (frequentemente chamado de spin, mas que vou evitar essa
terminologia pois ela está muito associado à mecânica quântica). Essa
solução descreve com grande acurácia algumas observações astronômicas, a
deflexão da luz pelo Sol, o periélio de Mercúrio, e de fato, por
décadas, essa solução era a única solução (exata) relativística que
mostrava como a Relatividade Geral tinha uma aplicabilidade maior que a
teoria Newtoniana. Além disso, a solução de Schwarzschild fornece um
modelo simples para alguns dos objetos mais misteriosos da natureza: Os
Buracos Negros.

Nessa aula, analisaremos a construção dessa solução, e começaremos a
analisar algumas de suas propriedades físicas mais importantes.

    \hypertarget{similaridades-com-o-eletromagnetismo}{%
\section{Similaridades com o
eletromagnetismo}\label{similaridades-com-o-eletromagnetismo}}

    Antes de começar a resolver a equação de Einstein, iremos fazer uma
breve (muito breve mesmo) revisão da lei de Gauss. Isso ajudará na
compreenssão de algumas escolhas que faremos no caso da resolução da
Equação de Einstein. Vamos começar relembrando a lei de Gauss cuja
expressão local é dada por
\[ \tag{2.1} \nabla \cdot \vec{E} = \frac{\rho}{\epsilon_0}\]

onde \(\rho\) é a densidade de cargas e é, em geral, uma função do tempo
e das coordenadas espaciais. Além disso, essa densidade é não nula num
volume \(V\) do espaço, é o que os matemáticos costumam chamar de
\emph{suporte finito}, ou seja,
\[\tag{2.2} \rho(t, x_1, x_2, x_3) = \left\{\begin{array}{lll}
 0 &se& (x_1, x_2, x_3) \notin V \\
 \neq 0 &se& (x_1, x_2, x_3) \in V \\
\end{array}\right. \]

Isso significa, em particular, que temos 2 regiões para se estudar essas
soluções, dentro e fora do corpo eletricamente carregado.As duas
soluções devem ser coladas de maneira apropriada com condições de
contorno apropriadas, e de fato, a solução interior é um pouco mais
sofisticada e por isso focaremos unicamente na solução exterior, onde a
densidade é nula.

A existência de simetrias simplifica bastante o estudo de sistemas que
satisfazem a lei de Gauss. O estudo de soluções exatas frequentemente
consideram a existência de algum tipo de simetria: por exemplo, podemos
assumir que o sistema não depende do tempo, e isso nos trás ao reino da
eletrostática. Ou ainda, podemos impor que o sistema tem simetria
esférica, planar e cilíndrica.

No caso da relatividade geral, simetria esférica é particularmente útil
pois `vistos de longe', diversos corpos celestes podem ser aproximados
por esferas perfeitas, planetas, estrelas, buracos negros e outros.
Claro que isso é algo de interesse astrofísico. Então vamos entendar
brevemente como esse tratamento é feito no caso eletrostático.

    \hypertarget{soluuxe7uxf5es-esfericamente-simuxe9tricas}{%
\subsection{Soluções Esfericamente
Simétricas}\label{soluuxe7uxf5es-esfericamente-simuxe9tricas}}

    Vamos considerar que o campo elétrico fora de um corpo esfericamente
simétrico. A primeira coisa que fazemos é escrever o nosso sistema em
coordenadas esféricas,
\[\tag{2.3} x_1 = r \sin\theta \cos\phi\quad x_2 = r \sin\theta \sin\phi\quad  x_3 = r \cos\theta\; ,\]
e além disso, essa simetria nos diz que o campo elétrico tem dependência
funcional apenas na direção radial, ou seja,
\[ \tag{2.4} \vec{E} = E(r) \hat{r}\] onde \(\hat{r}\) é o versor
radial.

Fora do corpo, a densidade de carga é zero, de tal forma que a lei de
Gauss é
\[ \tag{2.5} \nabla\cdot \vec{E} = \frac{1}{r^2}\frac{d}{dr}(r^2 E(r)) = 0 \quad \Rightarrow \quad E(r) = \frac{c_0}{r^2} \]
onde \(c_0\) é uma constante. Podemos determinar a constante \(c_0\)
integrando sobre o volume de uma esfera \(\mathbb{S}^2\) que engloba
todo o corpo carregado. Ou seja, usamos a lei de Gauss na forma
integral, pois é essa lei que nos informa das propriedades globais do
campo, ou seja
\[ \tag{2.6} \frac{Q}{\epsilon_0}= \int \nabla\cdot \vec{E} =  r^2 E(r) \int_{0}^\pi \sin\theta d\theta \int_{0}^{2\pi} = 4 \pi c_0  \]
onde \(Q\) é a carga total do corpo e na segunda igualdade usamos o
teorema da divergência. Assim, obtemos é o campo de Coulomb
\[ \tag{2.7} \vec{E} = \frac{Q}{4 \pi \epsilon_0 r^2} \hat{r} \; . \]

Empregaremos essa mesma lógica para resolver a Equação de Einstein no
caso de um corpo esfericamente simétrico.

    \hypertarget{simetria-de-gauge}{%
\subsection{Simetria de Gauge}\label{simetria-de-gauge}}

    Outro detalhe muito importante na teoria eletromagnética é que
(classicamente) os campos físicos mensuráveis são os campos elétrico
\(\vec{E}\) e magnético \(\vec{B}\), que escritos em termos dos
potenciais escalar \(\phi\) e vetor \(\vec{A}\) são
\[\tag{2.8} \vec{E} = - \nabla \phi - \frac{d}{dt}\vec{A} \qquad \vec{B} = \nabla \times \vec{A}  \]

E é simples verificar que os campos acima são invariantes pelas
transformações de gauge (ou calibre) dadas por
\[\tag{2.9} \phi \mapsto \phi' = \phi - \frac{d}{dt}\alpha \qquad \vec{A} \mapsto \vec{A}' = \vec{A} + \nabla \alpha \]
onde \(\alpha = \alpha(t, \vec{r})\) é uma função diferenciável
genérica. Como o parâmetro \(\alpha\) é uma função do espaço-tempo,
dizemos que essa é uma transformação local.

Tendo isso em mente, podemos reescrever as equações de Maxwell
unicamente em termos dos potenciais como (Ver seção 6.2 do Jackson)
\[\tag{2.10} \Delta \phi + \frac{\partial}{\partial t} (\nabla\cdot \vec{A}) = - \frac{\rho}{\epsilon_0} 
\qquad ; \qquad \Delta \vec{A} - \frac{1}{c^2} \frac{\partial^2 }{\partial t^2} \vec{A} - \nabla \left( \nabla \cdot \vec{A} + 
\frac{1}{c^2} \frac{\partial}{\partial t}\phi \right) = - \mu_0 \vec{J} \]

Mas como temos a simetria de Gauge, podemos escolher uma função
\(\alpha\) que force a seguinte condição
\[ \nabla \cdot \vec{A} + \frac{1}{c^2} \frac{\partial}{\partial t}\phi = 0 \]
que é chamado de Gauge de Lorenz. Infelizmente para Lorenz, o gauge
acima é invariante por transformações de Lorentz (e Lorentz é mais
famoso), então muitos chamam essa condição de Gauge de Lorentz. Agora é
simples observar que as equações (2.10) assumem a forma de equações de
onda inomogêneas
\[ \tag{2.11} \Delta \phi - \frac{1}{c^2} \frac{\partial}{\partial t} \phi = - \frac{\rho}{\epsilon_0}
\qquad \Delta \vec{A} - \frac{1}{c^2} \frac{\partial}{\partial t}  \vec{A} = - \mu_0 \vec{J}
\]

Outro Gauge particularmente útil é
\[ \tag{2.12} \nabla\cdot \vec{A} = 0 \] Chamado de Gauge de Coulomb,
pois o potencial escalar terá a forma que apendemos a amar no estudo da
eletrostática.

Duas são as conclusões importantes:

\begin{enumerate}
\def\labelenumi{\arabic{enumi})}
\item
  \emph{Apenas quantidades que são invariantes pelas transformações de
  gauge são quantidades mensuráveis.}
\item
  \emph{Podemos usar a simetria de gauge para simplificar o nosso
  problema.}
\end{enumerate}

Vamos manter esses dois fatos em mente.

    \hypertarget{ansuxe4tz-para-a-soluuxe7uxe3o-exterior-de-schwarzschild}{%
\section{Ansätz para a solução Exterior de
Schwarzschild}\label{ansuxe4tz-para-a-soluuxe7uxe3o-exterior-de-schwarzschild}}

    Vimos na primeira parte desse curso que a equação de campo de Einstein é
dada por:
\[ \tag{3.1} R_{\mu\nu} - \frac{1}{2} (R - 2 \Lambda) g_{\mu\nu} = \frac{8 \pi G}{c^4} T_{\mu\nu} \qquad (\mu, \nu = 1,2,3,4)\]
onde \(R_{\mu\nu}\) são as componentes do tensor de Ricci,
\(T_{\mu\nu}\) as componentes do tensor energia-momento, \(R\) o escalar
de curvatura, \(\Lambda\) a constante cosmológica. Além disso,
\(g_{\mu\nu}\) são as componentes da métrica que é um tensor simétrico,
e todas as componentes são funções das coordenadas do espaço-tempo, ou
seja \(g_{\mu\nu}=g_{\mu\nu}(x^0, x^1, x^2, x^3)\).

Por abuso de linguagem, vamos chamar o elemento de distântia \(ds^2\) de
métrica, e esse é dado por
\[ \tag{3.2} ds^2 = \sum_{\mu=0}^3 \sum_{\nu=0}^3 g_{\mu\nu} dx^\mu dx^\nu  \equiv  g_{\mu\nu} dx^\mu dx^\nu \]
onde na última igualdade usamos a convenção de Einstein para soma.

E assim como discutimos antes, iremos considerar uma solução
esfericamente simétrica na região exterior à fonte, ou seja temos que
resolver a equação de Einstein onde o tensor-energia momento
\(T_{\mu\nu} = 0\), ou seja
\[ \tag{3.3.a} R_{\mu\nu} - \frac{1}{2} (R - 2 \Lambda) g_{\mu\nu} = 0 \]

Além disso, usando que \(R = g^{\mu\nu} R_{\mu\nu}\) e que
\(g^{\mu\nu} g_{\mu\nu} = 4\), temos que a equação acima pode ser
escrita como

\[ \tag{3.3.b} R = 4\Lambda
\qquad \Leftrightarrow \qquad R_{\mu\nu} = \Lambda g_{\mu\nu}\]

E agora vamos começar a usar as nossas hipóteses simplificadoras.

    \hypertarget{simetria-esfuxe9rica}{%
\subsection{Simetria Esférica}\label{simetria-esfuxe9rica}}

    Nossa primeira hipótese é a simetria esférica da métrica. Vamos detalhar
um pouco essa hipótese agora. Vamos escrever todas as nossas equações em
termos de coordenadas esféricas. Já vimos na equação (2.3) que podemos
parametrizar a seção espacial em termos do raio e dos ângulos polar e
azimutal, dados respectivamente, por \(\theta \in [0, \pi)\) e
\(\phi \in [0, 2\pi)\).

Portanto escreveremos nossa métrica com índices na forma
\[ \tag{3.4} g = 
\begin{pmatrix}
g_{tt} & g_{tr}& g_{t\theta} & g_{t\phi} \\
g_{tr} & g_{rr} & g_{r\theta} & g_{r \phi} \\
g_{t\theta} & g_{r \theta} & g_{\theta \theta} & g_{\theta \phi} \\
g_{t\phi} & g_{r \phi} & g_{\theta\phi} & g_{\phi\phi} \\
\end{pmatrix} \]

    \hypertarget{esfera-simetria-so3}{%
\subsubsection{Esfera \& Simetria SO(3)}\label{esfera-simetria-so3}}

    Vamos primeiro trabalhar as coordenadas \((\theta, \phi)\), que definem
a esfera. Considere um raio fixo constante \(r_0\) constante,
\(\mathbb{S}^2\). Essa esfera é parametrizada por pontos no espaço
Euclidiano \(\mathbb{R}^3\) que satisfazem a condição
\[\tag{3.5} (x^1)^2 + (x^2)^2 + (x^3)^2 = r_0^2 \] Portanto
\[ \tag{3.6} x^1 = r_0 \sin\theta \cos\phi\quad x^2 = r_0 \sin\theta \sin\phi\quad  x^3 = r_0 \cos\theta\; .\]

Agora queremos calcular distâncias na esfera, ou seja, queremos
encontrar a métrica desse espaço. Como esse espaço está embebido no
espaço Euclidiano em 3D, que possui métrica
\[ ds^2_{Euc} = (dx^1)^2 + (dx^2)^2 + (dx^3)^2 \] ele herdará essa
métrica, ao impormos as condições (3.4) ou (3.5). Portanto
\[ \tag{3.7} ds^2_{\mathbb{S}^2} = (dx^1)^2 + (dx^2)^2 + (dx^3)^2 = r_0^2 (d\theta^2 + \sin^2\theta d\phi^2) \equiv r_0^2  d\Omega^2.\]

Isso significa que \[ \tag{3.8}  \boxed{g_{\theta \phi} = 0}\; \qquad 
\boxed{g_{\phi\phi}   = g_{\theta\theta} \sin^2\theta }. \] Além disso,
as componentes da métrica são funções do raio \(r\) e do tempo \(t\), ou
seja \[ \tag{3.9}  \boxed{g_{\mu\nu} = g_{\mu\nu}(t,r)}\]

    \hypertarget{foliauxe7uxe3o-do-espauxe7o-tempo}{%
\subsubsection{Foliação do
Espaço-Tempo}\label{foliauxe7uxe3o-do-espauxe7o-tempo}}

    No espaço Euclidiano, podemos usar (3.5) com \(r\) variável para
escrever a métrica coordenadas esféricas como
\[\tag{3.10} ds_{Euc}^2 = dr^2 + r^2 d\Omega^2\] E podemos dizer que
temos uma foliação do espaço Eucliano em esferas concentricas. Veja que
para cada raio \(r\) temos uma esfera diferente. Além disso, orientamos
o nosso sistema de coordenadas para que diferentes raios \(r_1\) e
\(r_2\) sejam identificados pelos mesmos ângulos \((\theta, \phi)\).
Essa é uma escolha bastante apropriada para o nosso problema, mas não é
sempre válida. Isso significa que as coordenadas radial e angulares
estão desacopladas. A consequência disso é que
\[ \tag{3.11} \boxed{g_{r \theta} = g_{r \phi} = 0}\; . \]

Além disso, supomos que uma foliação com relação no tempo, de forma que
para uma esfera de raio fixo, a evolução temporal nada mais será que uma
translação na esfera. Consequentemente
\[ \tag{3.12} \boxed{g_{t \theta} = g_{t \phi} = 0}\; . \]

    Para simplificar a nossa análise que faremos mais adiante, vamos
escrever \[ \tag{3.13} g_{\theta \theta} = F^2 r^2 \qquad F = F(t,r)\]

Portanto escreveremos nossa métrica com índices na forma
\[ \tag{3.14} g = 
\begin{pmatrix}
g_{tt} & g_{tr} & 0 & 0 \\
g_{tr} & g_{rr} & 0 & 0 \\
0 & 0 & (F r)^2  & 0 \\
0 & 0 & 0 & (F r)^2 \sin^2\theta \\
\end{pmatrix} \]

Além disso, não queremos que a assinatura da métrica mude, ou seja,
queremos que ela permaneça Lorentziana \(g_{tt} < 0\) e \(g_{rr} > 0\),
enquanto que \(g_{tr}\) não precisa ser especificada (ainda). Com isso,
o nosso ansätz é escrito como
\[ \tag{3.15} ds^2 = g_{tt} c^2 dt^2 + 2 g_{tr} c dt dr + g_{rr} dr^2 + (F r)^2 d \Omega^2  \]

    \hypertarget{transformauxe7uxf5es-gerais-de-coordenadas}{%
\subsubsection{Transformações Gerais de
Coordenadas}\label{transformauxe7uxf5es-gerais-de-coordenadas}}

    Sabemos pela discussão inicial do estudo da relatividade geral, que
também temos uma simetria local, a simetria de Difeomorfismo, que nós
físicos chamamos de \textbf{transformações gerais de coordenadas}. Isso
significa que podemos modificar as nossas coordenadas de tal forma que
as quantidades físicas relevantes são inalteradas, ou se transformam de
modo previsível. Isso é apenas o reflexo de que o Universo não vem
dotado de um sistema de coordenadas -- Isso é uma construção humana. No
entanto, existe uma noção de curvatura e quantidades relacionadas. Essa
é a ideia da relatividade geral. De toda forma, sistemas de coordenadas
relacionados por
\[ \tag{3.16} x^\mu \mapsto y^\sigma = y^\sigma(x) \qquad \Rightarrow \quad ds^2 = g_{\mu\nu} dx^\mu dx^\nu = g_{\mu\nu} \frac{\partial x^\mu}{\partial y^\sigma} \frac{\partial x^\nu}{\partial y^\lambda} dy^\sigma dy^\lambda \]
conduzem a mesma física. De fato, todos os objetos que consideramos até
agora são tensores com relação à essas transformações. Essas
transformações são análogas às simetrias de gauge que discutimos no
eletromagnetismo.

Na relatividade geral, temos a liberdade de fazer 4 redefinições de
coordenadas, o que nos conduzirá a 4 equações diferenciais aos moldes
dos \emph{gauge fixings}. Nosso objetivo será usar essas liberdades para
simplificar a métrica (3.15) ainda mais.

    \(\clubsuit\) \textbf{1.} Vamos considerar primeiramente uma
transformação de coordenadas na forma \[ \tag{3.16.a} t' = t'(t,r) \] de
forma que
\[ \tag{3.16.b} dt' = \frac{\partial t'}{\partial t} dt +  \frac{\partial t'}{\partial r} dr \]
e impomos a condição
\[ \tag{3.17} g_{tt} c^2 dt^2 + 2 g_{tr} dt dr + g_{rr} dr^2 = G_1 c^2 dt'^2 + G_2 dr^2 \]
o que significa que temos 3 ``\emph{condições de gauge}'' que determinam
as funções \(G_1(t,r)\) e \(G_2(t,r)\)
\[ \tag{3.18.a} g_{tt} = G_1 \left(\frac{\partial t'}{\partial t} \right)^2  \]
\[ \tag{3.18.a} g_{tr} = G_1 c \left(\frac{\partial t'}{\partial t} \right) \left(\frac{\partial t'}{\partial r} \right)  \]
\[ \tag{3.18.a} g_{rr} = G_1 c^2 \left(\frac{\partial t'}{\partial r} \right)^2 + G_2  \]
Agora assumimos que \(t'(t,r)\) é a nossa nova coordenada temporal.

O estudante mais atento pode se perguntar se podemos resolver esse
sistema de equações diferenciais. Primeiro, observe que temos 3 equações
para 3 funções desconhecidas: \(\{g_{tt}, g_{tr}, g_{rr}\}\). Além
disso, nossa hipótese é que as variedades que são fisicamente relevantes
são variedades suaves infinitamente diferenciáveis, portanto nós podemos
encontrar difeomosfismos entre abertos das variedades e o espaço de
Minkowski. Portanto: podemos (a princípio) resolver esse sistema.

Nesse novo sistema de coordenadas, a métrica torna-se
\[ \tag{3.19} ds^2 = G_1 dt'^2 + G_2 dr^2 + F r^2 d \Omega^2  \] que é
diagonal.

    \(\clubsuit\) \textbf{2.} Para usar a nossa liberdade de gauge
remanescente, vamos fazer uma escolha óbvia: Tomamos função
\(r'=r'(t,r)\) tal que \[ \tag{3.20} r' = F r\] E como acabamos de
dizer: Essa relação pode ser invertida, pois transformações gerais de
coordenadas é um outro nome para difeomorfismos, que são homeomorfismos
(bijeções contínuas) e diferenciáveis. Invertendo a relação acima temos
que \(r = f^{-1}(r')\), de forma que podemos escrever
\[\tag{3.21.a} r = \frac{r'}{F}\qquad \text{onde} \qquad F = F(t, f^{-1}(r')) \]
Mas veja que isso é um problema, pois devemos ter uma independencia
functional entre as coordenadas \(t\) e \(r\). Por isso, a função (F)
deve ser independente do tempo, ou seja \(F = F(r)\).

Assim
\[ \tag{3.21.b} dr' = \left(F + \frac{\partial F}{\partial r} \right) dr \qquad \Rightarrow \qquad dr = \left(F + \frac{\partial F}{\partial r} \right)^{-1} dr' = \left(F + \frac{\partial r'}{\partial r}\frac{\partial F}{\partial r'} \right)^{-1} dr'\]
de tal forma que
\[ \tag{3.22} ds^2 = G_1 c^2 dt'^2 + G_2 \left(F + \frac{\partial r'}{\partial r} \frac{\partial F}{\partial r'} \right)^{-2} dr'^2 + r'^2 d \Omega^2  \]

Portanto, usando as 4 condições de gauge conseguimos trazer a métrica
para uma forma muito mais simples. Porém está horrorosa.

    \hypertarget{ansuxe4tz-final}{%
\subsubsection{Ansätz final}\label{ansuxe4tz-final}}

    Tendo em vista os resultados acima, podemos finalmente chamar as
componentes da métrica (3.22) de outros nomes, para deixá-la numa forma
mais elegante. Primeiramente, podemos remover as linhas das novas
coordenadas temporal e radial, ou seja, escrevemos \(t' \equiv t\) e
\(r' \equiv r\).

Além disso, temos que lembrar que a métrica tem simetria Lorentziana.
Definimos agora duas funções \(\gamma_1(t,r)\) e \(\gamma_2(t,r)\) tar
que \[ \tag{3.23} G_1(t,r) = - e^{\gamma_1(t,r)} < 0 \qquad 
G_2 \left( F + \frac{\partial r'}{\partial r} \frac{\partial F}{\partial r'} \right)^{-2} = e^{\gamma_2(t,r)} > 0\]

Portanto, nosso ansätz é dado por
\[ \tag{3.24} \boxed{ds^2 = - e^{\gamma_1} c^2 dt^2 + e^{\gamma_2}dr^2 + r^2 (d\theta^2 + \sin^2\theta d\phi^2)}\]

Essa é a métrica que queremos considerar na equação de Einstein.

\textbf{Um ponto que deve ser destacado é que a coordenada \(r\) não é a
distância entre um ponto genérico e a origem da curvatura (ct,0,0,0).
Devemos pensar em \(r\) como uma coordenada definida de tal forma que ao
considerarmos ela fixa, a área da superfície definida por
\((\theta, \phi)\) é dada por \(4 \pi r^2\).}

    \hypertarget{resolvendo-a-equauxe7uxe3o-de-einstein}{%
\section{Resolvendo a Equação de
Einstein}\label{resolvendo-a-equauxe7uxe3o-de-einstein}}

    Primeiro vamos considerar o caso onde a constante cosmológica é zero,
portanto a equação de Einstein é dada por \[ \tag{4.1} R_{\mu\nu} = 0 \]
para o ansätz (3.24). No apêndice A, Todas as componentes do tensor de
Ricci são apresentadas. No apêndice B, usaremos esse mesmo ansätz para
estudar o caso onde a constante cosmológica é diferente de zero.

    Para a solução (3.24), temos 5 componentes não nulas no tensor de Ricci,
são elas
\[\tag{4.2} R_{tt}\qquad R_{tr}\qquad R_{rr}\qquad R_{\theta\theta} \qquad R_{\phi\phi} = \sin^2\theta R_{\theta\theta}\]

    \(\spadesuit\) \textbf{1.} A equação \(R_{tr} = 0\) é dada por
\[\tag{4.3} \frac{1}{r}\frac{\partial \gamma_2}{\partial t} = 0 \qquad \Rightarrow \qquad \gamma_2=f_2(r)\]
ou seja, a função \(f_2\) não depende do tempo.

    \(\spadesuit\) \textbf{2} Nesse ponto, podemos tomar \(f_2 = f_2(r)\) e
escrever as equações de Einstein

\begin{align*}
R_{tt} & = \frac{e^{f_1 - f_2}}{r}\left[\partial_r f_1 + \frac{1}{4}\left( (\partial_r f_1)^2 - \partial_r f_1 \partial_r f_2 +2 \partial^2_r f_1 \right) \right] = 0 \\
\tag{4.4} R_{rr} & = \partial_r f_2 + \frac{1}{4}\left( -(\partial_r f_1)^2 + \partial_r f_1 \partial_r f_2 - 2 \partial^2_r f_1 \right) = 0 \\
R_{\theta \theta} & = \frac{e^{-f_2}}{2}\left( - r \partial_r f_1 + r \partial_r f_2 + 2 e^{f_2} - 2 \right) = 0
\end{align*}

    \(\spadesuit\) \textbf{3.} Em seguida tomamos a derivada temporal de
\(R_{\theta\theta} = 0\). Isso implica a relação
\[\tag{4.5} -\partial_t R_{\theta\theta} = \frac{r e^{-\gamma_2}}{2}\frac{\partial^2 \gamma_1}{\partial r \partial t} = 0 \qquad \Rightarrow \qquad \gamma_1(r,t)=f_1(r) + 2g_1(t)\]
onde o coeficiente \(2\) é mera conveniência. Assim, a métrica se torna
\[ \tag{4.6} ds^2 = - e^{f_1} c^2 (e^{g_1(t)}dt)^2 + e^{f_2}dr^2 + r^2 (d\theta^2 + \sin^2\theta d\phi^2)\; .\]

Observe, no entando, que podemos voltar à nossas transformações gerais
de coordenadas (3.16.a), e adiantar para remover o termo inconveniente
na metrica (4.6), de tal forma que \(dt' = e^{g_1(t)}dt\) é a nossa
coordenada temporal. Observe que não estamos usando uma nova
transformação de coordenadas, estamos apenas melhorando a transformação
(3.16.a) para que ela remova esse termo.

Finalmente, temos que
\[ \tag{4.7} ds^2 = - e^{f_1} c^2 dt^2 + e^{f_2}dr^2 + r^2 (d\theta^2 + \sin^2\theta d\phi^2)\; ,\]
onde agora, todas as componentes da métrica são independentes do tempo,
e é dita \emph{estacionária}. Mas não apenas isso, essa métrica é também
\emph{estática}.

A diferença entre as duas definições é simples. Uma métrica é estática
se ela não se mexe, por exemplo, ela descreve um objeto que permanece
parado. Uma métrica é estacionária se ela não muda de uma forma
uniforme. Por exemplo, uma estrela girando a uma velocidade constante.
Uma métrica é estática se ela permanece igual se consideramos inversão
temporal \(t\mapsto -t\), e isso significa que métricas estáticas não
podem ter termos cruzados \(g_{t\mu}\) na métrica.

    \(\spadesuit\) \textbf{4.} Usando agora (4.4), observe que a estrutura
de \(R_tt\) e \(R_{rr}\), são bastante semelhantes. Portanto, temos
\[\tag{4.7} 0 = r(e^{-f_1 + f_2}R_{tt} + R_{rr}) = \frac{d}{dr}(f_1 + f_2)\qquad \Rightarrow \qquad  f_2 = -f_1 + c_0 \]
onde \(c_0\) é uma constante que tomamos, sem nenhuma perda de
generalidade, como zero.

    \(\spadesuit\) \textbf{5.} Finalmente, já sabemos que
\[ \tag{4.8} R_{\phi\phi} = \sin^2 \theta R_{\theta\theta}\] de forma
que nos resta apenas uma equações de Einstein para nos preocuparmos, que
é
\[ \tag{4.9} r e^{f_1} \frac{d f_1}{dr}  + e^{f_1} - 1 = 0 \qquad \Rightarrow \qquad \frac{r e^{f_1}}{dr} = 1  \]

    E é simples verificar que a solução para essa equação diferencial é

\[ \tag{4.10} r e^{f_1} = r + c_0 \qquad \Rightarrow \qquad  e^{f_1} = 1 + \frac{c_0}{r} \]

No exercício 2 na lista abaixo, o leitor deve provar que

\[ \tag{4.11} c_0 = - \frac{2 G M}{c^2}\; . \]

    \textbf{TEOREMA DE BIRKHOFF}

O teorema de Birkhoff afirma que o espaço-tempo de Schwarzschild é a
única solução que das equações de Einstein no vácuo que são estáticas,
esfericamente simétricas e assimptoticamente planas. É muito importante
entender que ignorando qualquer uma das condições acima, o teorema de
Birkhoff não será mais válido. Um exemplo a ser discutido nesse texto
como uma leitura extra é a solução de Sitter-Schwarzschild, que é a
generalização do espaço tempo Schwarzschild para o caso onde a constante
cosmológica não é nula.

A prova desse teorema é avançada para o nosso presente estudo, mas pode
ser encontrada com detalhes no livro Gravitation do
Mister-Thorn-Wheeler, seção 32.2.

    \hypertarget{leitura-extra-primeiro-encontro-com-os-buracos-negros}{%
\section{LEITURA EXTRA: Primeiro encontro com os buracos
negros}\label{leitura-extra-primeiro-encontro-com-os-buracos-negros}}

    Concluímos que a metrica é dada por

\[ \tag{5.1} ds^2 = -\left(1 - \frac{2 G M}{c^2 r}\right) c^2 dt^2 + \frac{dr^2}{\left(1 - \frac{2 G M}{c^2 r}\right)} + r^2 d\theta^2 + r^2 \sin^2\theta d\phi^2 \]

    A solução acima tem dois pontos onde a métrica diverge. O primeiro é
dado pelo ponto

\[ \tag{5.2} R_s = \frac{2GM}{c^2}\]

que é chamado de \textbf{Raio de Schwarzschild}, enquanto que o segundo
ponto é dado pelo ponto \(r=0\).

    Baseado na nossa comparação com o eletromagnetismo, já sabemos que
Transformações Gerais de Coordenadas são similares às simetrias de
gauge. Isso significa, também, que a métrica é um retalho (patch) do
espaço, e descreve localmente o que está acontecendo numa determinada
região do espaço-tempo, e que os resultados que obtemos nesse retalho
devem ser cuidadosamente comparadas com outros sistemas de coordenadas.

Para analisar esses pontos, devemos construir quantidades que não
dependam do sistema de coordenadas. Uma dessas quantidades é,
evidentemente, o escalar de curvatura \(R = g^{\mu\nu} R_{\mu\nu}\). No
caso dos espaços que satisfazem a equação (3.3.b), o escalar de
curvatura

\[ \tag{5.3} R = 4 \Lambda \]

O estudo de invariantes é fundamental na análise de divergencias na
métrica. Caso a divergência esteja presente no invariante, não importa o
sistema de coordenadas escolhidos, a divergência sempre aparecerá. Nesse
caso dizemos que o ponto é uma singularidade. Caso a singularidade não
esteja presente nos invariantes, então a divergência é um artefato do
sistema de coordenadas.

O espaço de Schwarzschild é uma solução plana \(R_{\mu\nu}=0\), portanto
o escalar de curvatura é nulo, portanto, não é um bom invariante.
Analisaremos o escalar de Kretschmann na seção abaixo.

Usando esse racional, podemos analisar a natureza dessa métrica
estudando o escalar de Kretschmann, que é dado por

\[ \tag{5.4} K = R^{\mu_0 \nu_0 \rho_0 \sigma_0} R_{\mu_0 \nu_0 \rho_0 \sigma_0} = g_{\mu_0 \mu_1} g^{\nu_0 \nu_1} g^{\rho_0 \rho_1} g^{\sigma_0 \sigma_1} R^{\mu_0}_{\nu_1 \rho_1 \sigma_1} R^{\mu_1}_{\nu_0 \rho_0\sigma_0} = \frac{48 G^2 M^2}{c^4 r^6} \]

    Observando o escalar de Kretschmann podemos ver que o raio de
Schwarzschild (5.2) é um ponto completamente ordinário, ao passo que o
ponto \(r=0\) é divergente. Alguns pontos importantes sobre essas
regiões são as seguintes:

    \textbf{\# RAIO DE SCHWARZSCHILD:} Essa hipersuperfície não define uma
singularidade física, mas é apenas um resquício do sistema de
coordenadas que estamos usando. Isso não significa, no entanto, que
esses pontos são desinteressantes. Na verdade, toda a superfície define
o horizonte de eventos. Em geral, o raio de Schwarzschild é muito menor
que o raio do corpo, por exemplo, para o Sol, o \(R_s \simeq 3 km\) ao
passo que para o planeta Terra, \(R_s \simeq 9 mm\). Ver tabela abaixo.

A métrica de Schwarzschild que discutimos nesse texto é uma solução
exterior, ou seja, é válida numa região onde o tensor energia momento é
nulo \(T_{\mu\nu} = 0\) o que também significa que \(r > R_s\). O estudo
da geometria interior exige outro sistema de coordenadas, chamada de
solução interior, é mais sofisticada e não discutiremos nessa aula (e
nem num curso de graduação). O que nos importa, no entando, é que
qualquer corpo que tenha um raio menor que o raio de Schwarzschild é
chamado de buraco negro. Evidente que toda essa discussão seria
puramente acadêmica se não existissem razões para acreditar na formação
desses objetos.

    \begin{tcolorbox}[breakable, size=fbox, boxrule=1pt, pad at break*=1mm,colback=cellbackground, colframe=cellborder]
\prompt{In}{incolor}{39}{\boxspacing}
\begin{Verbatim}[commandchars=\\\{\}]
\PY{c+c1}{\PYZsh{} Alguns valores do Raio de Schwarzschild}
\PY{n}{ScRadius} \PY{o}{=} \PY{n}{pd}\PY{o}{.}\PY{n}{read\PYZus{}html}\PY{p}{(}\PY{l+s+s1}{\PYZsq{}}\PY{l+s+s1}{https://en.wikipedia.org/wiki/Schwarzschild\PYZus{}radius}\PY{l+s+s1}{\PYZsq{}}\PY{p}{,} \PY{n}{match}\PY{o}{=}\PY{l+s+s2}{\PYZdq{}}\PY{l+s+s2}{Object}\PY{l+s+s2}{\PYZsq{}}\PY{l+s+s2}{s Schwarzschild radius}\PY{l+s+s2}{\PYZdq{}}\PY{p}{)} 
\PY{n}{table} \PY{o}{=} \PY{n}{ScRadius}\PY{p}{[}\PY{l+m+mi}{0}\PY{p}{]}\PY{p}{[}\PY{p}{[}\PY{l+s+s1}{\PYZsq{}}\PY{l+s+s1}{Object}\PY{l+s+s1}{\PYZsq{}}\PY{p}{,}\PY{l+s+s1}{\PYZsq{}}\PY{l+s+s1}{Mass}\PY{l+s+s1}{\PYZsq{}}\PY{p}{,}\PY{l+s+s1}{\PYZsq{}}\PY{l+s+s1}{Schwarzschild radius}\PY{l+s+s1}{\PYZsq{}}\PY{p}{]}\PY{p}{]}\PY{o}{.}\PY{n}{loc}\PY{p}{[}\PY{l+m+mi}{10}\PY{p}{:}\PY{l+m+mi}{20}\PY{p}{]}
\end{Verbatim}
\end{tcolorbox}

    \begin{tcolorbox}[breakable, size=fbox, boxrule=1pt, pad at break*=1mm,colback=cellbackground, colframe=cellborder]
\prompt{In}{incolor}{40}{\boxspacing}
\begin{Verbatim}[commandchars=\\\{\}]
\PY{n}{table}
\end{Verbatim}
\end{tcolorbox}

            \begin{tcolorbox}[breakable, size=fbox, boxrule=.5pt, pad at break*=1mm, opacityfill=0]
\prompt{Out}{outcolor}{40}{\boxspacing}
\begin{Verbatim}[commandchars=\\\{\}]
     Object           Mass Schwarzschild radius
10      Sun   1.99×1030 kg           2.95×103 m
11  Jupiter   1.90×1027 kg               2.82 m
12   Saturn  5.683×1026 kg          8.42×10−1 m
13  Neptune  1.024×1026 kg          1.52×10−1 m
14   Uranus  8.681×1025 kg          1.29×10−1 m
15    Earth   5.97×1024 kg          8.87×10−3 m
16    Venus  4.867×1024 kg          7.21×10−3 m
17     Mars   6.39×1023 kg          9.47×10−4 m
18  Mercury  3.285×1023 kg          4.87×10−4 m
19     Moon   7.35×1022 kg          1.09×10−4 m
20    Human          70 kg         1.04×10−25 m
\end{Verbatim}
\end{tcolorbox}
        
    \textbf{\# SINGULARIDADE:} O ponto \(r=0\) é uma singularidade física e
não será removida pela escolha de um novo sistema de coordenada. O fato
de existir essa singularidade foi razão para que muitos físicos
acreditassem que buracos negros não poderiam existir. Roger Penrose
provou uma série de teoremas que mostram que singularidades desse tipo
(e consequente os Buracos Negros) não são objetos extraordinários na
Relatividade Geral. Antes de continuarmos, é importante mencionar que a
maioria dos físicos acreditam que efeitos quânticos de alguma formam
controlam a formação dessas singularidades.

    \textbf{\# FORMAÇÃO DE UM BURACO NEGRO:} A estabilidade de uma estrela é
determinada pelo equilíbrio entre a pressão da atração gravitacional e a
das reações termonucleares, sobretudo no processo de fusão nuclear.
Quando o combustível começa a acabar, a estrela começa a esfriar, a
atração gravitacional tende a superar as reações nucleares, e para a
maioria das estrelas, restará apenas um objeto estelar ultra-denso
chamado de \emph{Anã Branca}. Esse será o destino do nosso querido Sol,
que formará uma Anã Branca de raio \(\simeq 5000 km\) (muito maior que
seu \(R_s\)), mas com uma densidade de \(10^9 kg /m^3\). O mecanismo que
controla esse colapso gravitacional das Anãs Brancas é a pressão de
degenerescência dos elétrons. Nesse caso, os elétrons se comportam como
um gás de Fermi, e eles ocupam todos os estados com energia abaixo da
Energia de Fermi, de tal forma que esse o princípio de exclusão de Pauli
salva a estrela de um destino mais trágico.

Chandrasekhar provou (antes de Penrose) que se a corpo estelar
remanescente (chamado de Remnant em inglês) tiver \(1.4 M_{\odot}\), que
hoje chamamos de Limite de Chandrasekhar (onde \(M_{\odot}\) é a massa
solar), a pressão de degenerescência não é suficiente para controlar o
colapso gravitacional. Com a descoberta dos neutrons por James Chadwick
em 1932, ficou claro embora a pressão de degenerescência fosse
insuficiente para controlar o colapso gravitacional, os elétrons
interagem com os prótrons formando neutrons (e neutrinos) pelo inverso
do decaimento beta. Isso forma uma estrela de Neutrons. A título de
comparação, se o nosso Sol virasse uma estrela de neutros, ele teria um
raio de apenas \(30km\). Esses são os objetos mais densos na Natureza
depois dos buracos negros, que são formados quando a massa do objeto
estelar é da ordem de \(3M_{\odot}\), chamado de limite de
Tolman--Oppenheimer--Volkoff.

    \hypertarget{exercuxedcios}{%
\section{Exercícios}\label{exercuxedcios}}

    Os problemas marcados com uma estrela são muito importantes e devem ser
respondidos.

\textbf{Problema 1.} Usando as coordenadas esféricas
\[ x_1 = \sin\theta \cos\phi\quad 
x_2 = \sin\theta \sin\phi\quad  
x_3 = \cos\theta\] prove que a métrica no espaço Euclidiano em 3D é dada
por \[ ds^2_{Euc} = dr^2 + r^2 d\Omega^2 \] onde a métrica da esfera
unitária \(\mathbb{S}^2\) é
\[ d\Omega^2 = d\theta^2 + \sin^2 \theta d\phi^2 \]

\textbf{Problema 2.} Usando a métrica da esfera unitária, mostre que os
vetores de Killing da esfera \(\mathbb{S}^2\) são \[ 
\xi^{(1)} = \sin\phi \partial_\theta + \cot\theta\cos\phi \partial_\phi\quad 
\xi^{(2)} = \cos\phi \partial_\theta - \cot\theta\sin\phi \partial_\phi \quad
\xi^{(3)} = \partial_\phi
\] e que eles satisfazem
\[ [\xi^{(a)}, \xi^{(b)}] = \epsilon^{abc}\xi^{(c)}  \] que é a álgebra
de Lie \(SO(3)\). Sabendo que a solução de Schwarzschild tem 4 vetores
de Killing, use esse resultado e escreva todos os campos vetoriais de
Killing explicitamente.

\(\star\) \textbf{Problema 3.} Usando as equações de Einstein (4.4),
prove as relações (4.5), (4.7) e (4.10).

\(\star\) \textbf{Problema 4.} Vimos na primeira parte desse curso que
no limite Newtoniano, o potencial gravitacional é relacionado à
componente \(g_{00}\) por
\(g_{00} = -\left(1 + 2\frac{\phi}{c^2}\right)\). Repita o argumento que
fizemos na primeira parte desse curso, reconsiderando uma partícula
livre numa geometria definida pela métrica
\[ ds^2 = -\left(1 + \frac{c_0}{r}\right) c^2 dt^2 + \frac{dr^2}{\left(1 + \frac{c_0}{r}\right)} + r^2 d\Omega_2^2 \]

De fato, estudando a equação da geodésica para a solução de
Schwarzschild (que encontraremos abaixo) nos levará ao potencial
\(\phi(r) = - \frac{GM}{r}\). Comparando esse resultado com a lei de
Gauss gravitacional
\[\nabla\cdot \vec{E}_g = -\Delta\phi = - 4 \pi G \rho  \quad \Leftrightarrow \quad 
\int_{\partial V} d \vec{S}\cdot \vec{E}_g  = -4 \pi G M \] onde
\(\vec{E}_g\) é o campo gravitacional e \(\rho\) é a densidade de massa,
use os mesmos argumentos do eletromagnetismo para justificar o resultado
\[ c_0 = - \frac{2GM}{c^2} \]

\textbf{Problema 5.} Alternativamente, podemos buscar uma solução na
forma
\[ ds^2 = - a(\rho)^2 c^2 dt^2 + b(\rho)^2 (d\rho^2 + \rho^2(d\theta^2 + \sin^2\theta d\phi^2)) \]
onde \(\rho=\rho(r)\) é uma nova direção radial. Essas soluções são
chamadas de \emph{isotrópicas}, e para \(t=constante\), o espaço-tempo
nessas coordenadas é conforme ao espaço Euclidiano em 3D. Veja que as
direções angulares não são modificadas.

Usando uma tranformação da forma
\[ r = \rho\left( 1 + \frac{c_0}{2 \rho} \right)^2\; , \] determine a
constante \(c_0\) e as funções \(a(\rho)\) e \(b(\rho)\) para a solução
de Schwarzschild. Onde se encontra o Horizonte de Eventos?

    \hypertarget{resolvendo-a-equauxe7uxe3o-de-einstein-com-o-python}{%
\section{Resolvendo a Equação de Einstein com o
Python}\label{resolvendo-a-equauxe7uxe3o-de-einstein-com-o-python}}

    Vamos agora resolver a solução de Einstein passo a passo usando o ansätz
dado na expressão (3).

    \begin{tcolorbox}[breakable, size=fbox, boxrule=1pt, pad at break*=1mm,colback=cellbackground, colframe=cellborder]
\prompt{In}{incolor}{1}{\boxspacing}
\begin{Verbatim}[commandchars=\\\{\}]
\PY{c+c1}{\PYZsh{} METADADOS \PYZam{} PACOTES}
\PY{k+kn}{from} \PY{n+nn}{sympy} \PY{k+kn}{import} \PY{o}{*}
\PY{k+kn}{from} \PY{n+nn}{itertools} \PY{k+kn}{import} \PY{o}{*}
\PY{k+kn}{import} \PY{n+nn}{pandas} \PY{k}{as} \PY{n+nn}{pd}

\PY{k+kn}{import} \PY{n+nn}{matplotlib}\PY{n+nn}{.}\PY{n+nn}{pyplot} \PY{k}{as} \PY{n+nn}{plt}
\PY{k+kn}{import} \PY{n+nn}{numpy} \PY{k}{as} \PY{n+nn}{np}
\PY{c+c1}{\PYZsh{} Einsteinpy: https://docs.einsteinpy.org/en/stable/index.html}
\PY{c+c1}{\PYZsh{} Sympy: https://docs.sympy.org/latest/index.html}
\PY{c+c1}{\PYZsh{} Intertools: https://docs.python.org/3/library/itertools.html}
\end{Verbatim}
\end{tcolorbox}

    \begin{tcolorbox}[breakable, size=fbox, boxrule=1pt, pad at break*=1mm,colback=cellbackground, colframe=cellborder]
\prompt{In}{incolor}{3}{\boxspacing}
\begin{Verbatim}[commandchars=\\\{\}]
\PY{c+c1}{\PYZsh{} PARÂMETROS }
\PY{c+c1}{\PYZsh{} Massa M, constante de Newton G, velocidade da luz c, constante cosmológica Lb, tempo t, raio r, angulo polar theta th, azimuth phi ph }
\PY{n}{M}\PY{p}{,} \PY{n}{G}\PY{p}{,} \PY{n}{c}\PY{p}{,} \PY{n}{Lb}\PY{p}{,} \PY{n}{t}\PY{p}{,} \PY{n}{r}\PY{p}{,} \PY{n}{th}\PY{p}{,} \PY{n}{ph} \PY{o}{=} \PY{n}{symbols}\PY{p}{(}\PY{l+s+s1}{\PYZsq{}}\PY{l+s+s1}{M G c Lb t r th ph}\PY{l+s+s1}{\PYZsq{}}\PY{p}{)}
\PY{n}{X} \PY{o}{=} \PY{p}{[}\PY{n}{t}\PY{p}{,} \PY{n}{r}\PY{p}{,} \PY{n}{th}\PY{p}{,} \PY{n}{ph}\PY{p}{]}
\PY{c+c1}{\PYZsh{} Também definimos as funções f1 e f2 que usamos no ansätz para a solução de Schwarzschild}
\PY{n}{f1} \PY{o}{=} \PY{n}{Function}\PY{p}{(}\PY{l+s+s1}{\PYZsq{}}\PY{l+s+s1}{f1}\PY{l+s+s1}{\PYZsq{}}\PY{p}{)}\PY{p}{(}\PY{l+s+s1}{\PYZsq{}}\PY{l+s+s1}{r}\PY{l+s+s1}{\PYZsq{}}\PY{p}{)} 
\PY{n}{f2} \PY{o}{=} \PY{n}{Function}\PY{p}{(}\PY{l+s+s1}{\PYZsq{}}\PY{l+s+s1}{f2}\PY{l+s+s1}{\PYZsq{}}\PY{p}{)}\PY{p}{(}\PY{l+s+s1}{\PYZsq{}}\PY{l+s+s1}{r}\PY{l+s+s1}{\PYZsq{}}\PY{p}{)} 

\PY{c+c1}{\PYZsh{} Para verificar as soluções explícitas, use}
\PY{c+c1}{\PYZsh{} SCHWARZSCHILD: }
\PY{c+c1}{\PYZsh{}f1 = ln(1 \PYZhy{} 2*G*M / (r*c**2)) }
\PY{c+c1}{\PYZsh{}f2 = \PYZhy{} f1}
\PY{c+c1}{\PYZsh{} De Sitter: }
\PY{c+c1}{\PYZsh{}f1 = ln(1 \PYZhy{} 2*G*M / (r*c**2) \PYZhy{} Lb * r**2  / 3)}
\PY{c+c1}{\PYZsh{}f2 = \PYZhy{}f1 }
\end{Verbatim}
\end{tcolorbox}

    \hypertarget{muxe9trica}{%
\subsection{Métrica}\label{muxe9trica}}

    \begin{tcolorbox}[breakable, size=fbox, boxrule=1pt, pad at break*=1mm,colback=cellbackground, colframe=cellborder]
\prompt{In}{incolor}{4}{\boxspacing}
\begin{Verbatim}[commandchars=\\\{\}]
\PY{c+c1}{\PYZsh{} Aqui vamos definir a métrica. Inicialmente preenchemos todas as entradas com zeros, e depois definimos as componentes não nulas. }
\PY{n}{g} \PY{o}{=} \PY{n}{zeros}\PY{p}{(}\PY{l+m+mi}{4}\PY{p}{)} 
\PY{n}{g}\PY{p}{[}\PY{l+m+mi}{0}\PY{p}{,}\PY{l+m+mi}{0}\PY{p}{]} \PY{o}{=} \PY{o}{\PYZhy{}}\PY{n}{exp}\PY{p}{(}\PY{n}{f1}\PY{p}{)} \PY{c+c1}{\PYZsh{} felizmente o python é relativista e começa a contar os indices a partir do zero }
\PY{n}{g}\PY{p}{[}\PY{l+m+mi}{1}\PY{p}{,}\PY{l+m+mi}{1}\PY{p}{]} \PY{o}{=} \PY{n}{exp}\PY{p}{(}\PY{n}{f2}\PY{p}{)}
\PY{n}{g}\PY{p}{[}\PY{l+m+mi}{2}\PY{p}{,}\PY{l+m+mi}{2}\PY{p}{]} \PY{o}{=} \PY{n}{r}\PY{o}{*}\PY{o}{*}\PY{l+m+mi}{2}
\PY{n}{g}\PY{p}{[}\PY{l+m+mi}{3}\PY{p}{,}\PY{l+m+mi}{3}\PY{p}{]} \PY{o}{=} \PY{n}{r}\PY{o}{*}\PY{o}{*}\PY{l+m+mi}{2} \PY{o}{*} \PY{n}{sin}\PY{p}{(}\PY{n}{th}\PY{p}{)}\PY{o}{*}\PY{o}{*}\PY{l+m+mi}{2}

\PY{n}{rank} \PY{o}{=} \PY{n}{g}\PY{o}{.}\PY{n}{rank}\PY{p}{(}\PY{p}{)} \PY{c+c1}{\PYZsh{} o rank será útil para generalizar esse código para casos onde o espaço\PYZhy{}tempo tem dimensões maiores ou menos que 4. Sim, isso é muito importante. }
\PY{n}{gI} \PY{o}{=} \PY{n}{g}\PY{o}{.}\PY{n}{inv}\PY{p}{(}\PY{p}{)} \PY{c+c1}{\PYZsh{} Também precisamos da inversa para subir os indices de Lorentz}
\end{Verbatim}
\end{tcolorbox}

    \hypertarget{suxedmbolos-de-christoffel}{%
\subsection{Símbolos de Christoffel}\label{suxedmbolos-de-christoffel}}

    Esses objetos são as componentes da conexão de Levi-Civita dadas
explicitamente por:

\begin{equation}
\Gamma^\rho_{\mu\nu} = \frac{1}{2} g^{\rho\sigma} \left( \partial_\mu g_{\sigma\nu} + \partial_\sigma g_{\mu\nu} - \partial_\sigma g_{\mu\nu} \right)
\end{equation}

    \begin{tcolorbox}[breakable, size=fbox, boxrule=1pt, pad at break*=1mm,colback=cellbackground, colframe=cellborder]
\prompt{In}{incolor}{5}{\boxspacing}
\begin{Verbatim}[commandchars=\\\{\}]
\PY{c+c1}{\PYZsh{} k denota os indices contravariantes (no andar de cima), ao passo que i e j denotam os indices covariantes (no andar de baixo).}
\PY{k}{def} \PY{n+nf}{Gamma}\PY{p}{(}\PY{n}{k}\PY{p}{,}\PY{n}{i}\PY{p}{,}\PY{n}{j}\PY{p}{)}\PY{p}{:}
    \PY{n}{n} \PY{o}{=} \PY{l+m+mi}{0}
    \PY{n}{GammaMatrix} \PY{o}{=} \PY{l+m+mi}{0}
    \PY{k}{while} \PY{n}{n} \PY{o}{\PYZlt{}} \PY{n}{rank}\PY{p}{:}
        \PY{n}{GammaMatrix} \PY{o}{=} \PY{n}{GammaMatrix} \PY{o}{+} \PY{n}{gI}\PY{p}{[}\PY{n}{k}\PY{p}{,}\PY{n}{n}\PY{p}{]}\PY{o}{*}\PY{p}{(}\PY{n}{g}\PY{p}{[}\PY{n}{n}\PY{p}{,}\PY{n}{j}\PY{p}{]}\PY{o}{.}\PY{n}{diff}\PY{p}{(}\PY{n}{X}\PY{p}{[}\PY{n}{i}\PY{p}{]}\PY{p}{)} \PY{o}{+} \PY{n}{g}\PY{p}{[}\PY{n}{n}\PY{p}{,}\PY{n}{i}\PY{p}{]}\PY{o}{.}\PY{n}{diff}\PY{p}{(}\PY{n}{X}\PY{p}{[}\PY{n}{j}\PY{p}{]}\PY{p}{)} \PY{o}{\PYZhy{}} \PY{n}{g}\PY{p}{[}\PY{n}{i}\PY{p}{,}\PY{n}{j}\PY{p}{]}\PY{o}{.}\PY{n}{diff}\PY{p}{(}\PY{n}{X}\PY{p}{[}\PY{n}{n}\PY{p}{]}\PY{p}{)}\PY{p}{)} \PY{o}{/} \PY{l+m+mi}{2}
        \PY{n}{n} \PY{o}{+}\PY{o}{=} \PY{l+m+mi}{1}
    \PY{k}{return} \PY{n}{GammaMatrix}
\end{Verbatim}
\end{tcolorbox}

    É útil pensar nesses objetos, os símbolos de Christoffel como 4 matrizes
rotulados com o indice k = 0, 1, 2, 3, de tal forma que:

    \begin{tcolorbox}[breakable, size=fbox, boxrule=1pt, pad at break*=1mm,colback=cellbackground, colframe=cellborder]
\prompt{In}{incolor}{6}{\boxspacing}
\begin{Verbatim}[commandchars=\\\{\}]
\PY{c+c1}{\PYZsh{} Primeiro construímos 4 matrizes com todas as componentes nulas}
\PY{n}{ChSymb0} \PY{o}{=} \PY{n}{zeros}\PY{p}{(}\PY{n}{rank}\PY{p}{)}
\PY{n}{ChSymb1} \PY{o}{=} \PY{n}{zeros}\PY{p}{(}\PY{n}{rank}\PY{p}{)}
\PY{n}{ChSymb2} \PY{o}{=} \PY{n}{zeros}\PY{p}{(}\PY{n}{rank}\PY{p}{)}
\PY{n}{ChSymb3} \PY{o}{=} \PY{n}{zeros}\PY{p}{(}\PY{n}{rank}\PY{p}{)}
\PY{n}{ChSymb} \PY{o}{=} \PY{p}{[}\PY{n}{ChSymb0}\PY{p}{,} \PY{n}{ChSymb1}\PY{p}{,} \PY{n}{ChSymb2}\PY{p}{,} \PY{n}{ChSymb3}\PY{p}{]}
\end{Verbatim}
\end{tcolorbox}

    \begin{tcolorbox}[breakable, size=fbox, boxrule=1pt, pad at break*=1mm,colback=cellbackground, colframe=cellborder]
\prompt{In}{incolor}{7}{\boxspacing}
\begin{Verbatim}[commandchars=\\\{\}]
\PY{c+c1}{\PYZsh{} Agora preenchemos essas matrizes com as componentes não nulas para o ansätz dado em (3)}
\PY{k}{for} \PY{n}{k} \PY{o+ow}{in} \PY{n+nb}{range}\PY{p}{(}\PY{n}{rank}\PY{p}{)}\PY{p}{:}
    \PY{k}{for} \PY{n}{i} \PY{o+ow}{in} \PY{n+nb}{range}\PY{p}{(}\PY{n}{rank}\PY{p}{)}\PY{p}{:}
        \PY{k}{for} \PY{n}{j} \PY{o+ow}{in} \PY{n+nb}{range}\PY{p}{(}\PY{n}{rank}\PY{p}{)}\PY{p}{:}
            \PY{k}{if} \PY{n}{Gamma}\PY{p}{(}\PY{n}{k}\PY{p}{,}\PY{n}{i}\PY{p}{,}\PY{n}{j}\PY{p}{)} \PY{o}{!=} \PY{l+m+mi}{0}\PY{p}{:}
                \PY{n}{ChSymb}\PY{p}{[}\PY{n}{k}\PY{p}{]}\PY{p}{[}\PY{n}{i}\PY{p}{,}\PY{n}{j}\PY{p}{]} \PY{o}{=} \PY{n}{Gamma}\PY{p}{(}\PY{n}{k}\PY{p}{,}\PY{n}{i}\PY{p}{,}\PY{n}{j}\PY{p}{)}
\end{Verbatim}
\end{tcolorbox}

    \begin{tcolorbox}[breakable, size=fbox, boxrule=1pt, pad at break*=1mm,colback=cellbackground, colframe=cellborder]
\prompt{In}{incolor}{24}{\boxspacing}
\begin{Verbatim}[commandchars=\\\{\}]
\PY{c+c1}{\PYZsh{} Para ver o resultado explícito, fazemos}
\PY{k}{for} \PY{n}{k} \PY{o+ow}{in} \PY{n+nb}{range}\PY{p}{(}\PY{n+nb}{len}\PY{p}{(}\PY{n}{ChSymb}\PY{p}{)}\PY{p}{)}\PY{p}{:}
    \PY{n+nb}{print}\PY{p}{(}\PY{l+s+sa}{f}\PY{l+s+s2}{\PYZdq{}}\PY{l+s+s2}{+++++++++ Christoffel(}\PY{l+s+si}{\PYZob{}}\PY{n}{k}\PY{l+s+si}{\PYZcb{}}\PY{l+s+s2}{) ++++++++++++++++}\PY{l+s+s2}{\PYZdq{}}\PY{p}{)}
    \PY{n}{display}\PY{p}{(}\PY{n}{ChSymb}\PY{p}{[}\PY{n}{k}\PY{p}{]}\PY{p}{)}
    \PY{n+nb}{print}\PY{p}{(}\PY{l+s+s2}{\PYZdq{}}\PY{l+s+s2}{ }\PY{l+s+s2}{\PYZdq{}}\PY{p}{)}
\end{Verbatim}
\end{tcolorbox}

    \begin{Verbatim}[commandchars=\\\{\}]
+++++++++ Christoffel(0) ++++++++++++++++
    \end{Verbatim}

    $\displaystyle \left[\begin{matrix}0 & \frac{\frac{d}{d r} f_{1}{\left(r \right)}}{2} & 0 & 0\\\frac{\frac{d}{d r} f_{1}{\left(r \right)}}{2} & 0 & 0 & 0\\0 & 0 & 0 & 0\\0 & 0 & 0 & 0\end{matrix}\right]$

    
    \begin{Verbatim}[commandchars=\\\{\}]

+++++++++ Christoffel(1) ++++++++++++++++
    \end{Verbatim}

    $\displaystyle \left[\begin{matrix}\frac{e^{2 f_{1}{\left(r \right)}} \frac{d}{d r} f_{1}{\left(r \right)}}{2} & 0 & 0 & 0\\0 & - \frac{\frac{d}{d r} f_{1}{\left(r \right)}}{2} & 0 & 0\\0 & 0 & - r e^{f_{1}{\left(r \right)}} & 0\\0 & 0 & 0 & - r e^{f_{1}{\left(r \right)}} \sin^{2}{\left(th \right)}\end{matrix}\right]$

    
    \begin{Verbatim}[commandchars=\\\{\}]

+++++++++ Christoffel(2) ++++++++++++++++
    \end{Verbatim}

    $\displaystyle \left[\begin{matrix}0 & 0 & 0 & 0\\0 & 0 & \frac{1}{r} & 0\\0 & \frac{1}{r} & 0 & 0\\0 & 0 & 0 & - \sin{\left(th \right)} \cos{\left(th \right)}\end{matrix}\right]$

    
    \begin{Verbatim}[commandchars=\\\{\}]

+++++++++ Christoffel(3) ++++++++++++++++
    \end{Verbatim}

    $\displaystyle \left[\begin{matrix}0 & 0 & 0 & 0\\0 & 0 & 0 & \frac{1}{r}\\0 & 0 & 0 & \frac{\cos{\left(th \right)}}{\sin{\left(th \right)}}\\0 & \frac{1}{r} & \frac{\cos{\left(th \right)}}{\sin{\left(th \right)}} & 0\end{matrix}\right]$

    
    \begin{Verbatim}[commandchars=\\\{\}]

    \end{Verbatim}

    \hypertarget{tensor-de-riemann}{%
\subsection{Tensor de Riemann}\label{tensor-de-riemann}}

    De posse dos simbolos de Christoffel, podemos encontrar o tensor de
Riemann pela expressão

\begin{equation}
R^{\lambda}_{\rho \mu \nu} = \partial_\mu \Gamma^{\lambda}_{\nu \rho} - \partial_\lambda \Gamma^\nu_{\rho \mu} + \Gamma^\lambda_{\mu\sigma}\Gamma^\sigma_{\rho\nu} -  \Gamma^\lambda_{\nu \sigma}\Gamma^\sigma_{\rho\mu}
\end{equation}

    \begin{tcolorbox}[breakable, size=fbox, boxrule=1pt, pad at break*=1mm,colback=cellbackground, colframe=cellborder]
\prompt{In}{incolor}{8}{\boxspacing}
\begin{Verbatim}[commandchars=\\\{\}]
\PY{c+c1}{\PYZsh{} l é contravariante; k,i,j são covariantes}
\PY{c+c1}{\PYZsh{} Aqui vamos definir a constração do produto de dois simbolos de Christoffel. Ou seja, os dois últimos termos na equação (5)}
\PY{k}{def} \PY{n+nf}{Contraction}\PY{p}{(}\PY{n}{l}\PY{p}{,}\PY{n}{k}\PY{p}{,}\PY{n}{i}\PY{p}{,}\PY{n}{j}\PY{p}{)}\PY{p}{:}
    \PY{n}{n} \PY{o}{=} \PY{l+m+mi}{0} 
    \PY{n}{cont} \PY{o}{=} \PY{l+m+mi}{0} 
    \PY{k}{while} \PY{n}{n} \PY{o}{\PYZlt{}} \PY{n}{rank}\PY{p}{:}
        \PY{n}{cont} \PY{o}{=} \PY{n}{cont} \PY{o}{+} \PY{n}{ChSymb}\PY{p}{[}\PY{n}{l}\PY{p}{]}\PY{p}{[}\PY{n}{i}\PY{p}{,}\PY{n}{n}\PY{p}{]} \PY{o}{*} \PY{n}{ChSymb}\PY{p}{[}\PY{n}{n}\PY{p}{]}\PY{p}{[}\PY{n}{k}\PY{p}{,}\PY{n}{j}\PY{p}{]} \PY{o}{\PYZhy{}} \PY{n}{ChSymb}\PY{p}{[}\PY{n}{l}\PY{p}{]}\PY{p}{[}\PY{n}{j}\PY{p}{,}\PY{n}{n}\PY{p}{]} \PY{o}{*} \PY{n}{ChSymb}\PY{p}{[}\PY{n}{n}\PY{p}{]}\PY{p}{[}\PY{n}{k}\PY{p}{,}\PY{n}{i}\PY{p}{]}
        \PY{n}{n} \PY{o}{+}\PY{o}{=} \PY{l+m+mi}{1}
    \PY{k}{return} \PY{n}{cont}
\end{Verbatim}
\end{tcolorbox}

    \begin{tcolorbox}[breakable, size=fbox, boxrule=1pt, pad at break*=1mm,colback=cellbackground, colframe=cellborder]
\prompt{In}{incolor}{9}{\boxspacing}
\begin{Verbatim}[commandchars=\\\{\}]
\PY{c+c1}{\PYZsh{} Agora calculamos as componentes do tensor de Riemann propriamente ditas. }
\PY{k}{def} \PY{n+nf}{Riem}\PY{p}{(}\PY{n}{l}\PY{p}{,}\PY{n}{k}\PY{p}{,}\PY{n}{i}\PY{p}{,}\PY{n}{j}\PY{p}{)}\PY{p}{:}
    \PY{n}{n} \PY{o}{=} \PY{l+m+mi}{0}
    \PY{n}{Riemann} \PY{o}{=} \PY{l+m+mi}{0}
    \PY{k}{while} \PY{n}{n} \PY{o}{\PYZlt{}} \PY{n}{rank}\PY{p}{:}
        \PY{n}{Riemann} \PY{o}{=} \PY{n}{Riemann} \PY{o}{+} \PY{n}{ChSymb}\PY{p}{[}\PY{n}{l}\PY{p}{]}\PY{p}{[}\PY{n}{k}\PY{p}{,}\PY{n}{j}\PY{p}{]}\PY{o}{.}\PY{n}{diff}\PY{p}{(}\PY{n}{X}\PY{p}{[}\PY{n}{i}\PY{p}{]}\PY{p}{)} \PY{o}{\PYZhy{}} \PY{n}{ChSymb}\PY{p}{[}\PY{n}{l}\PY{p}{]}\PY{p}{[}\PY{n}{k}\PY{p}{,}\PY{n}{i}\PY{p}{]}\PY{o}{.}\PY{n}{diff}\PY{p}{(}\PY{n}{X}\PY{p}{[}\PY{n}{j}\PY{p}{]}\PY{p}{)} \PY{o}{+} \PY{n}{Contraction}\PY{p}{(}\PY{n}{l}\PY{p}{,}\PY{n}{k}\PY{p}{,}\PY{n}{i}\PY{p}{,}\PY{n}{j}\PY{p}{)}
        \PY{n}{n} \PY{o}{+}\PY{o}{=} \PY{l+m+mi}{1}
    \PY{n}{R} \PY{o}{=} \PY{n}{Riemann}\PY{o}{.}\PY{n}{simplify}\PY{p}{(}\PY{p}{)}
    \PY{k}{return} \PY{n}{R}
\end{Verbatim}
\end{tcolorbox}

    Com as componentes que podemos calcular usando a função acima, vamos
organizar esses resultados em termos de matrizes
\((R^\lambda_\rho)_{\mu\nu}\)

    \begin{tcolorbox}[breakable, size=fbox, boxrule=1pt, pad at break*=1mm,colback=cellbackground, colframe=cellborder]
\prompt{In}{incolor}{10}{\boxspacing}
\begin{Verbatim}[commandchars=\\\{\}]
\PY{c+c1}{\PYZsh{} Mais uma vez, definimos as matrizes e preenchemos com zeros. }
\PY{n}{RieTen00} \PY{o}{=} \PY{n}{zeros}\PY{p}{(}\PY{n}{rank}\PY{p}{)}\PY{p}{;} \PY{n}{RieTen01} \PY{o}{=} \PY{n}{zeros}\PY{p}{(}\PY{n}{rank}\PY{p}{)}\PY{p}{;} \PY{n}{RieTen02} \PY{o}{=} \PY{n}{zeros}\PY{p}{(}\PY{n}{rank}\PY{p}{)}\PY{p}{;} \PY{n}{RieTen03} \PY{o}{=} \PY{n}{zeros}\PY{p}{(}\PY{n}{rank}\PY{p}{)}
\PY{n}{RieTen10} \PY{o}{=} \PY{n}{zeros}\PY{p}{(}\PY{n}{rank}\PY{p}{)}\PY{p}{;} \PY{n}{RieTen11} \PY{o}{=} \PY{n}{zeros}\PY{p}{(}\PY{n}{rank}\PY{p}{)}\PY{p}{;} \PY{n}{RieTen12} \PY{o}{=} \PY{n}{zeros}\PY{p}{(}\PY{n}{rank}\PY{p}{)}\PY{p}{;} \PY{n}{RieTen13} \PY{o}{=} \PY{n}{zeros}\PY{p}{(}\PY{n}{rank}\PY{p}{)}
\PY{n}{RieTen20} \PY{o}{=} \PY{n}{zeros}\PY{p}{(}\PY{n}{rank}\PY{p}{)}\PY{p}{;} \PY{n}{RieTen21} \PY{o}{=} \PY{n}{zeros}\PY{p}{(}\PY{n}{rank}\PY{p}{)}\PY{p}{;} \PY{n}{RieTen22} \PY{o}{=} \PY{n}{zeros}\PY{p}{(}\PY{n}{rank}\PY{p}{)}\PY{p}{;} \PY{n}{RieTen23} \PY{o}{=} \PY{n}{zeros}\PY{p}{(}\PY{n}{rank}\PY{p}{)}
\PY{n}{RieTen30} \PY{o}{=} \PY{n}{zeros}\PY{p}{(}\PY{n}{rank}\PY{p}{)}\PY{p}{;} \PY{n}{RieTen31} \PY{o}{=} \PY{n}{zeros}\PY{p}{(}\PY{n}{rank}\PY{p}{)}\PY{p}{;} \PY{n}{RieTen32} \PY{o}{=} \PY{n}{zeros}\PY{p}{(}\PY{n}{rank}\PY{p}{)}\PY{p}{;} \PY{n}{RieTen33} \PY{o}{=} \PY{n}{zeros}\PY{p}{(}\PY{n}{rank}\PY{p}{)}

\PY{n}{Rie} \PY{o}{=} \PY{p}{[}\PY{p}{[}\PY{n}{RieTen00}\PY{p}{,} \PY{n}{RieTen01}\PY{p}{,} \PY{n}{RieTen02}\PY{p}{,} \PY{n}{RieTen03}\PY{p}{]}\PY{p}{,} 
       \PY{p}{[}\PY{n}{RieTen10}\PY{p}{,} \PY{n}{RieTen11}\PY{p}{,} \PY{n}{RieTen12}\PY{p}{,} \PY{n}{RieTen13}\PY{p}{]}\PY{p}{,}
       \PY{p}{[}\PY{n}{RieTen20}\PY{p}{,} \PY{n}{RieTen21}\PY{p}{,} \PY{n}{RieTen22}\PY{p}{,} \PY{n}{RieTen23}\PY{p}{]}\PY{p}{,}
       \PY{p}{[}\PY{n}{RieTen30}\PY{p}{,} \PY{n}{RieTen31}\PY{p}{,} \PY{n}{RieTen32}\PY{p}{,} \PY{n}{RieTen33}\PY{p}{]}\PY{p}{]}
\end{Verbatim}
\end{tcolorbox}

    \begin{tcolorbox}[breakable, size=fbox, boxrule=1pt, pad at break*=1mm,colback=cellbackground, colframe=cellborder]
\prompt{In}{incolor}{11}{\boxspacing}
\begin{Verbatim}[commandchars=\\\{\}]
\PY{c+c1}{\PYZsh{} E agora podemos preencher as componentes não nulas nas matrizes acima}
\PY{k}{for} \PY{n}{l} \PY{o+ow}{in} \PY{n+nb}{range}\PY{p}{(}\PY{n}{rank}\PY{p}{)}\PY{p}{:}
    \PY{k}{for} \PY{n}{k} \PY{o+ow}{in} \PY{n+nb}{range}\PY{p}{(}\PY{n}{rank}\PY{p}{)}\PY{p}{:}
        \PY{k}{for} \PY{n}{i} \PY{o+ow}{in} \PY{n+nb}{range}\PY{p}{(}\PY{n}{rank}\PY{p}{)}\PY{p}{:}
            \PY{k}{for} \PY{n}{j} \PY{o+ow}{in} \PY{n+nb}{range}\PY{p}{(}\PY{n}{rank}\PY{p}{)}\PY{p}{:}
                \PY{k}{if} \PY{n}{Riem}\PY{p}{(}\PY{n}{l}\PY{p}{,}\PY{n}{k}\PY{p}{,}\PY{n}{i}\PY{p}{,}\PY{n}{j}\PY{p}{)} \PY{o}{!=} \PY{l+m+mi}{0}\PY{p}{:}
                    \PY{n}{Rie}\PY{p}{[}\PY{n}{l}\PY{p}{]}\PY{p}{[}\PY{n}{k}\PY{p}{]}\PY{p}{[}\PY{n}{i}\PY{p}{,}\PY{n}{j}\PY{p}{]} \PY{o}{=} \PY{n}{Riem}\PY{p}{(}\PY{n}{l}\PY{p}{,}\PY{n}{k}\PY{p}{,}\PY{n}{i}\PY{p}{,}\PY{n}{j}\PY{p}{)} \PY{o}{/} \PY{l+m+mi}{4} 
\PY{+w}{                    }\PY{l+s+sd}{\PYZsq{}\PYZsq{}\PYZsq{}}
\PY{l+s+sd}{                    Aqui temos que explicar a presença dessa divisão por 4. No meu código, estamos separando um tensor em várias matrizes, }
\PY{l+s+sd}{                    e isso introduz redundâncias devido aos loops que estamos tomando sobre esses objetos. }
\PY{l+s+sd}{                    \PYZsq{}\PYZsq{}\PYZsq{}}
\end{Verbatim}
\end{tcolorbox}

    \begin{tcolorbox}[breakable, size=fbox, boxrule=1pt, pad at break*=1mm,colback=cellbackground, colframe=cellborder]
\prompt{In}{incolor}{29}{\boxspacing}
\begin{Verbatim}[commandchars=\\\{\}]
\PY{c+c1}{\PYZsh{} Para ver o resultado explícito, fazemos}
\PY{k}{for} \PY{n}{l} \PY{o+ow}{in} \PY{n+nb}{range}\PY{p}{(}\PY{n}{rank}\PY{p}{)}\PY{p}{:}
    \PY{k}{for} \PY{n}{k} \PY{o+ow}{in} \PY{n+nb}{range}\PY{p}{(}\PY{n}{rank}\PY{p}{)}\PY{p}{:}
        \PY{n}{display}\PY{p}{(}\PY{n}{Rie}\PY{p}{[}\PY{n}{l}\PY{p}{]}\PY{p}{[}\PY{n}{k}\PY{p}{]}\PY{p}{)}
\end{Verbatim}
\end{tcolorbox}

    $\displaystyle \left[\begin{matrix}0 & 0 & 0 & 0\\0 & 0 & 0 & 0\\0 & 0 & 0 & 0\\0 & 0 & 0 & 0\end{matrix}\right]$

    
    $\displaystyle \left[\begin{matrix}0 & - \frac{\left(\frac{d}{d r} f_{1}{\left(r \right)}\right)^{2}}{2} - \frac{\frac{d^{2}}{d r^{2}} f_{1}{\left(r \right)}}{2} & 0 & 0\\\frac{\left(\frac{d}{d r} f_{1}{\left(r \right)}\right)^{2}}{2} + \frac{\frac{d^{2}}{d r^{2}} f_{1}{\left(r \right)}}{2} & 0 & 0 & 0\\0 & 0 & 0 & 0\\0 & 0 & 0 & 0\end{matrix}\right]$

    
    $\displaystyle \left[\begin{matrix}0 & 0 & - \frac{r e^{f_{1}{\left(r \right)}} \frac{d}{d r} f_{1}{\left(r \right)}}{2} & 0\\0 & 0 & 0 & 0\\\frac{r e^{f_{1}{\left(r \right)}} \frac{d}{d r} f_{1}{\left(r \right)}}{2} & 0 & 0 & 0\\0 & 0 & 0 & 0\end{matrix}\right]$

    
    $\displaystyle \left[\begin{matrix}0 & 0 & 0 & - \frac{r e^{f_{1}{\left(r \right)}} \sin^{2}{\left(th \right)} \frac{d}{d r} f_{1}{\left(r \right)}}{2}\\0 & 0 & 0 & 0\\0 & 0 & 0 & 0\\\frac{r e^{f_{1}{\left(r \right)}} \sin^{2}{\left(th \right)} \frac{d}{d r} f_{1}{\left(r \right)}}{2} & 0 & 0 & 0\end{matrix}\right]$

    
    $\displaystyle \left[\begin{matrix}0 & \frac{\left(- \left(\frac{d}{d r} f_{1}{\left(r \right)}\right)^{2} - \frac{d^{2}}{d r^{2}} f_{1}{\left(r \right)}\right) e^{2 f_{1}{\left(r \right)}}}{2} & 0 & 0\\\frac{\left(\left(\frac{d}{d r} f_{1}{\left(r \right)}\right)^{2} + \frac{d^{2}}{d r^{2}} f_{1}{\left(r \right)}\right) e^{2 f_{1}{\left(r \right)}}}{2} & 0 & 0 & 0\\0 & 0 & 0 & 0\\0 & 0 & 0 & 0\end{matrix}\right]$

    
    $\displaystyle \left[\begin{matrix}0 & 0 & 0 & 0\\0 & 0 & 0 & 0\\0 & 0 & 0 & 0\\0 & 0 & 0 & 0\end{matrix}\right]$

    
    $\displaystyle \left[\begin{matrix}0 & 0 & 0 & 0\\0 & 0 & - \frac{r e^{f_{1}{\left(r \right)}} \frac{d}{d r} f_{1}{\left(r \right)}}{2} & 0\\0 & \frac{r e^{f_{1}{\left(r \right)}} \frac{d}{d r} f_{1}{\left(r \right)}}{2} & 0 & 0\\0 & 0 & 0 & 0\end{matrix}\right]$

    
    $\displaystyle \left[\begin{matrix}0 & 0 & 0 & 0\\0 & 0 & 0 & - \frac{r e^{f_{1}{\left(r \right)}} \sin^{2}{\left(th \right)} \frac{d}{d r} f_{1}{\left(r \right)}}{2}\\0 & 0 & 0 & 0\\0 & \frac{r e^{f_{1}{\left(r \right)}} \sin^{2}{\left(th \right)} \frac{d}{d r} f_{1}{\left(r \right)}}{2} & 0 & 0\end{matrix}\right]$

    
    $\displaystyle \left[\begin{matrix}0 & 0 & - \frac{e^{2 f_{1}{\left(r \right)}} \frac{d}{d r} f_{1}{\left(r \right)}}{2 r} & 0\\0 & 0 & 0 & 0\\\frac{e^{2 f_{1}{\left(r \right)}} \frac{d}{d r} f_{1}{\left(r \right)}}{2 r} & 0 & 0 & 0\\0 & 0 & 0 & 0\end{matrix}\right]$

    
    $\displaystyle \left[\begin{matrix}0 & 0 & 0 & 0\\0 & 0 & \frac{\frac{d}{d r} f_{1}{\left(r \right)}}{2 r} & 0\\0 & - \frac{\frac{d}{d r} f_{1}{\left(r \right)}}{2 r} & 0 & 0\\0 & 0 & 0 & 0\end{matrix}\right]$

    
    $\displaystyle \left[\begin{matrix}0 & 0 & 0 & 0\\0 & 0 & 0 & 0\\0 & 0 & 0 & 0\\0 & 0 & 0 & 0\end{matrix}\right]$

    
    $\displaystyle \left[\begin{matrix}0 & 0 & 0 & 0\\0 & 0 & 0 & 0\\0 & 0 & 0 & \left(1 - e^{f_{1}{\left(r \right)}}\right) \sin^{2}{\left(th \right)}\\0 & 0 & \left(e^{f_{1}{\left(r \right)}} - 1\right) \sin^{2}{\left(th \right)} & 0\end{matrix}\right]$

    
    $\displaystyle \left[\begin{matrix}0 & 0 & 0 & - \frac{e^{2 f_{1}{\left(r \right)}} \frac{d}{d r} f_{1}{\left(r \right)}}{2 r}\\0 & 0 & 0 & 0\\0 & 0 & 0 & 0\\\frac{e^{2 f_{1}{\left(r \right)}} \frac{d}{d r} f_{1}{\left(r \right)}}{2 r} & 0 & 0 & 0\end{matrix}\right]$

    
    $\displaystyle \left[\begin{matrix}0 & 0 & 0 & 0\\0 & 0 & 0 & \frac{\frac{d}{d r} f_{1}{\left(r \right)}}{2 r}\\0 & 0 & 0 & 0\\0 & - \frac{\frac{d}{d r} f_{1}{\left(r \right)}}{2 r} & 0 & 0\end{matrix}\right]$

    
    $\displaystyle \left[\begin{matrix}0 & 0 & 0 & 0\\0 & 0 & 0 & 0\\0 & 0 & 0 & e^{f_{1}{\left(r \right)}} - 1\\0 & 0 & 1 - e^{f_{1}{\left(r \right)}} & 0\end{matrix}\right]$

    
    $\displaystyle \left[\begin{matrix}0 & 0 & 0 & 0\\0 & 0 & 0 & 0\\0 & 0 & 0 & 0\\0 & 0 & 0 & 0\end{matrix}\right]$

    
    \hypertarget{tensor-de-ricci}{%
\subsection{Tensor de Ricci}\label{tensor-de-ricci}}

    Finalmente, podemos calcular o tensor de Ricci como \begin{equation}
R_{\mu\nu} = R^\rho_{\mu \rho \nu}
\end{equation}

    \begin{tcolorbox}[breakable, size=fbox, boxrule=1pt, pad at break*=1mm,colback=cellbackground, colframe=cellborder]
\prompt{In}{incolor}{12}{\boxspacing}
\begin{Verbatim}[commandchars=\\\{\}]
\PY{k}{def} \PY{n+nf}{Ricci}\PY{p}{(}\PY{n}{i}\PY{p}{,}\PY{n}{j}\PY{p}{)}\PY{p}{:}
    \PY{n}{ricc} \PY{o}{=} \PY{l+m+mi}{0}
    \PY{n}{n} \PY{o}{=} \PY{l+m+mi}{0}
    \PY{k}{for} \PY{n}{k} \PY{o+ow}{in} \PY{n+nb}{range}\PY{p}{(}\PY{n}{rank}\PY{p}{)}\PY{p}{:}
        \PY{n}{ricc} \PY{o}{=} \PY{n}{ricc} \PY{o}{+} \PY{n}{Rie}\PY{p}{[}\PY{n}{k}\PY{p}{]}\PY{p}{[}\PY{n}{i}\PY{p}{]}\PY{p}{[}\PY{n}{k}\PY{p}{,}\PY{n}{j}\PY{p}{]}
        \PY{n}{r} \PY{o}{=} \PY{n}{ricc}\PY{o}{.}\PY{n}{simplify}\PY{p}{(}\PY{p}{)}
    \PY{k}{return} \PY{n}{r}
\end{Verbatim}
\end{tcolorbox}

    Agora vamos organizar esse tensor na forma matricial

    \begin{tcolorbox}[breakable, size=fbox, boxrule=1pt, pad at break*=1mm,colback=cellbackground, colframe=cellborder]
\prompt{In}{incolor}{13}{\boxspacing}
\begin{Verbatim}[commandchars=\\\{\}]
\PY{n}{Ric} \PY{o}{=} \PY{n}{zeros}\PY{p}{(}\PY{n}{rank}\PY{p}{)}
\end{Verbatim}
\end{tcolorbox}

    \begin{tcolorbox}[breakable, size=fbox, boxrule=1pt, pad at break*=1mm,colback=cellbackground, colframe=cellborder]
\prompt{In}{incolor}{14}{\boxspacing}
\begin{Verbatim}[commandchars=\\\{\}]
\PY{c+c1}{\PYZsh{} E agora podemos preencher as componentes não nulas nas matrizes acima}
\PY{k}{for} \PY{n}{i} \PY{o+ow}{in} \PY{n+nb}{range}\PY{p}{(}\PY{n}{rank}\PY{p}{)}\PY{p}{:}
    \PY{k}{for} \PY{n}{j} \PY{o+ow}{in} \PY{n+nb}{range}\PY{p}{(}\PY{n}{rank}\PY{p}{)}\PY{p}{:}
         \PY{k}{if} \PY{n}{Ricci}\PY{p}{(}\PY{n}{i}\PY{p}{,}\PY{n}{j}\PY{p}{)} \PY{o}{!=} \PY{l+m+mi}{0}\PY{p}{:}
            \PY{n}{Ric}\PY{p}{[}\PY{n}{i}\PY{p}{,}\PY{n}{j}\PY{p}{]} \PY{o}{=} \PY{n}{Ricci}\PY{p}{(}\PY{n}{i}\PY{p}{,}\PY{n}{j}\PY{p}{)}
\end{Verbatim}
\end{tcolorbox}

    \begin{tcolorbox}[breakable, size=fbox, boxrule=1pt, pad at break*=1mm,colback=cellbackground, colframe=cellborder]
\prompt{In}{incolor}{33}{\boxspacing}
\begin{Verbatim}[commandchars=\\\{\}]
\PY{c+c1}{\PYZsh{} Para ver o resultado explícito, fazemos}
\PY{k}{for} \PY{n}{i} \PY{o+ow}{in} \PY{n+nb}{range}\PY{p}{(}\PY{n}{rank}\PY{p}{)}\PY{p}{:}
    \PY{k}{for} \PY{n}{j} \PY{o+ow}{in} \PY{n+nb}{range}\PY{p}{(}\PY{n}{rank}\PY{p}{)}\PY{p}{:}
        \PY{k}{if} \PY{n}{Ric}\PY{p}{[}\PY{n}{i}\PY{p}{,}\PY{n}{j}\PY{p}{]} \PY{o}{!=} \PY{l+m+mi}{0}\PY{p}{:}
            \PY{n+nb}{print}\PY{p}{(}\PY{l+s+sa}{f}\PY{l+s+s2}{\PYZdq{}}\PY{l+s+s2}{+++++++++ Ricci(}\PY{l+s+si}{\PYZob{}}\PY{n}{i}\PY{l+s+si}{\PYZcb{}}\PY{l+s+s2}{,}\PY{l+s+si}{\PYZob{}}\PY{n}{j}\PY{l+s+si}{\PYZcb{}}\PY{l+s+s2}{) ++++++++++++++++}\PY{l+s+s2}{\PYZdq{}}\PY{p}{)}
            \PY{n}{display}\PY{p}{(}\PY{n}{Ric}\PY{p}{[}\PY{n}{i}\PY{p}{,}\PY{n}{j}\PY{p}{]}\PY{p}{)}
            \PY{n+nb}{print}\PY{p}{(}\PY{l+s+s2}{\PYZdq{}}\PY{l+s+s2}{ }\PY{l+s+s2}{\PYZdq{}}\PY{p}{)}
\end{Verbatim}
\end{tcolorbox}

    \begin{Verbatim}[commandchars=\\\{\}]
+++++++++ Ricci(0,0) ++++++++++++++++
    \end{Verbatim}

    $\displaystyle \frac{\left(\frac{r \left(\left(\frac{d}{d r} f_{1}{\left(r \right)}\right)^{2} + \frac{d^{2}}{d r^{2}} f_{1}{\left(r \right)}\right)}{2} + \frac{d}{d r} f_{1}{\left(r \right)}\right) e^{2 f_{1}{\left(r \right)}}}{r}$

    
    \begin{Verbatim}[commandchars=\\\{\}]

+++++++++ Ricci(1,1) ++++++++++++++++
    \end{Verbatim}

    $\displaystyle \frac{\frac{r \left(- \left(\frac{d}{d r} f_{1}{\left(r \right)}\right)^{2} - \frac{d^{2}}{d r^{2}} f_{1}{\left(r \right)}\right)}{2} - \frac{d}{d r} f_{1}{\left(r \right)}}{r}$

    
    \begin{Verbatim}[commandchars=\\\{\}]

+++++++++ Ricci(2,2) ++++++++++++++++
    \end{Verbatim}

    $\displaystyle - r e^{f_{1}{\left(r \right)}} \frac{d}{d r} f_{1}{\left(r \right)} - e^{f_{1}{\left(r \right)}} + 1$

    
    \begin{Verbatim}[commandchars=\\\{\}]

+++++++++ Ricci(3,3) ++++++++++++++++
    \end{Verbatim}

    $\displaystyle \left(- r e^{f_{1}{\left(r \right)}} \frac{d}{d r} f_{1}{\left(r \right)} - e^{f_{1}{\left(r \right)}} + 1\right) \sin^{2}{\left(th \right)}$

    
    \begin{Verbatim}[commandchars=\\\{\}]

    \end{Verbatim}

    \(\clubsuit\) \textbf{1.} Apenas 3 equações são independentes. É simples
verificar que \begin{equation}
R_{33} - sin^2\theta = 0 
\end{equation}

    \begin{tcolorbox}[breakable, size=fbox, boxrule=1pt, pad at break*=1mm,colback=cellbackground, colframe=cellborder]
\prompt{In}{incolor}{15}{\boxspacing}
\begin{Verbatim}[commandchars=\\\{\}]
\PY{p}{(}\PY{n}{Ric}\PY{p}{[}\PY{l+m+mi}{3}\PY{p}{,}\PY{l+m+mi}{3}\PY{p}{]} \PY{o}{\PYZhy{}} \PY{n}{Ric}\PY{p}{[}\PY{l+m+mi}{2}\PY{p}{,}\PY{l+m+mi}{2}\PY{p}{]}\PY{o}{*}\PY{p}{(}\PY{n}{sin}\PY{p}{(}\PY{n}{th}\PY{p}{)}\PY{o}{*}\PY{o}{*}\PY{l+m+mi}{2}\PY{p}{)}\PY{p}{)}\PY{o}{.}\PY{n}{simplify}\PY{p}{(}\PY{p}{)}
\end{Verbatim}
\end{tcolorbox}
 
            
\prompt{Out}{outcolor}{15}{}
    
    $\displaystyle 0$

    

    \(\clubsuit\) \textbf{2.} As 2 primeiras equações possuem uma estrutura
muito semelhante, então faremos a operação a seguir

\begin{equation}
r(e^{-f_1 + f_2} R_{00} + R_{11})  = 0
\end{equation}

    \begin{tcolorbox}[breakable, size=fbox, boxrule=1pt, pad at break*=1mm,colback=cellbackground, colframe=cellborder]
\prompt{In}{incolor}{16}{\boxspacing}
\begin{Verbatim}[commandchars=\\\{\}]
\PY{n}{r}\PY{o}{*}\PY{p}{(}\PY{n}{exp}\PY{p}{(}\PY{o}{\PYZhy{}}\PY{n}{f1}\PY{o}{+}\PY{n}{f2}\PY{p}{)}\PY{o}{*}\PY{n}{Ric}\PY{p}{[}\PY{l+m+mi}{0}\PY{p}{,}\PY{l+m+mi}{0}\PY{p}{]} \PY{o}{+} \PY{n}{Ric}\PY{p}{[}\PY{l+m+mi}{1}\PY{p}{,}\PY{l+m+mi}{1}\PY{p}{]}\PY{p}{)}\PY{o}{.}\PY{n}{simplify}\PY{p}{(}\PY{p}{)}
\end{Verbatim}
\end{tcolorbox}
 
            
\prompt{Out}{outcolor}{16}{}
    
    $\displaystyle \frac{d}{d r} f_{1}{\left(r \right)} + \frac{d}{d r} f_{2}{\left(r \right)}$

    

    de onde concluímos que \begin{equation}
f_1 + f_2 = c_0 \quad \Rightarrow \quad f_2 = -f_1  
\end{equation} onde \(c_0\) é uma constante que escolhemos como zero.

    \(\clubsuit\) \textbf{3.} Agora vamos usar essa condição na equação
\begin{equation}
R_{22}\ = 0
\end{equation}

    \begin{tcolorbox}[breakable, size=fbox, boxrule=1pt, pad at break*=1mm,colback=cellbackground, colframe=cellborder]
\prompt{In}{incolor}{34}{\boxspacing}
\begin{Verbatim}[commandchars=\\\{\}]
\PY{n}{Ric}\PY{p}{[}\PY{l+m+mi}{2}\PY{p}{,}\PY{l+m+mi}{2}\PY{p}{]}\PY{o}{.}\PY{n}{simplify}\PY{p}{(}\PY{p}{)}
\end{Verbatim}
\end{tcolorbox}
 
            
\prompt{Out}{outcolor}{34}{}
    
    $\displaystyle - r e^{f_{1}{\left(r \right)}} \frac{d}{d r} f_{1}{\left(r \right)} - e^{f_{1}{\left(r \right)}} + 1$

    

    \hypertarget{escalar-de-kretschmann}{%
\subsection{Escalar de Kretschmann}\label{escalar-de-kretschmann}}

    \begin{tcolorbox}[breakable, size=fbox, boxrule=1pt, pad at break*=1mm,colback=cellbackground, colframe=cellborder]
\prompt{In}{incolor}{17}{\boxspacing}
\begin{Verbatim}[commandchars=\\\{\}]
\PY{n}{indices} \PY{o}{=} \PY{n+nb}{list}\PY{p}{(}\PY{n}{product}\PY{p}{(}\PY{p}{[}\PY{l+m+mi}{0}\PY{p}{,}\PY{l+m+mi}{1}\PY{p}{,}\PY{l+m+mi}{2}\PY{p}{,}\PY{l+m+mi}{3}\PY{p}{]}\PY{p}{,} \PY{n}{repeat}\PY{o}{=}\PY{l+m+mi}{2}\PY{p}{)}\PY{p}{)}
\PY{k}{def} \PY{n+nf}{Kr}\PY{p}{(}\PY{p}{)}\PY{p}{:}
    \PY{n}{K} \PY{o}{=} \PY{l+m+mi}{0}
    \PY{k}{for} \PY{n}{m} \PY{o+ow}{in} \PY{n}{indices}\PY{p}{:}
        \PY{k}{for} \PY{n}{n} \PY{o+ow}{in} \PY{n}{indices}\PY{p}{:}
            \PY{k}{for} \PY{n}{p} \PY{o+ow}{in} \PY{n}{indices}\PY{p}{:} 
                \PY{k}{for} \PY{n}{q} \PY{o+ow}{in} \PY{n}{indices}\PY{p}{:}
                    \PY{n}{K} \PY{o}{+}\PY{o}{=} \PY{n}{g}\PY{p}{[}\PY{n}{m}\PY{p}{[}\PY{l+m+mi}{0}\PY{p}{]}\PY{p}{,}\PY{n}{m}\PY{p}{[}\PY{l+m+mi}{1}\PY{p}{]}\PY{p}{]} \PY{o}{*} \PY{n}{gI}\PY{p}{[}\PY{n}{n}\PY{p}{[}\PY{l+m+mi}{0}\PY{p}{]}\PY{p}{,}\PY{n}{n}\PY{p}{[}\PY{l+m+mi}{1}\PY{p}{]}\PY{p}{]} \PY{o}{*} \PY{n}{gI}\PY{p}{[}\PY{n}{p}\PY{p}{[}\PY{l+m+mi}{0}\PY{p}{]}\PY{p}{,}\PY{n}{p}\PY{p}{[}\PY{l+m+mi}{1}\PY{p}{]}\PY{p}{]} \PY{o}{*} \PY{n}{gI}\PY{p}{[}\PY{n}{q}\PY{p}{[}\PY{l+m+mi}{0}\PY{p}{]}\PY{p}{,}\PY{n}{q}\PY{p}{[}\PY{l+m+mi}{1}\PY{p}{]}\PY{p}{]} \PY{o}{*} \PY{n}{Rie}\PY{p}{[}\PY{n}{m}\PY{p}{[}\PY{l+m+mi}{0}\PY{p}{]}\PY{p}{]}\PY{p}{[}\PY{n}{n}\PY{p}{[}\PY{l+m+mi}{1}\PY{p}{]}\PY{p}{]}\PY{p}{[}\PY{n}{p}\PY{p}{[}\PY{l+m+mi}{1}\PY{p}{]}\PY{p}{,} \PY{n}{q}\PY{p}{[}\PY{l+m+mi}{1}\PY{p}{]}\PY{p}{]} \PY{o}{*} \PY{n}{Rie}\PY{p}{[}\PY{n}{m}\PY{p}{[}\PY{l+m+mi}{1}\PY{p}{]}\PY{p}{]}\PY{p}{[}\PY{n}{n}\PY{p}{[}\PY{l+m+mi}{0}\PY{p}{]}\PY{p}{]}\PY{p}{[}\PY{n}{p}\PY{p}{[}\PY{l+m+mi}{0}\PY{p}{]}\PY{p}{,}\PY{n}{q}\PY{p}{[}\PY{l+m+mi}{0}\PY{p}{]}\PY{p}{]}
    \PY{n}{K} \PY{o}{=} \PY{n}{K}\PY{o}{.}\PY{n}{simplify}\PY{p}{(}\PY{p}{)}
    \PY{k}{return} \PY{n}{K}
\end{Verbatim}
\end{tcolorbox}

    \begin{tcolorbox}[breakable, size=fbox, boxrule=1pt, pad at break*=1mm,colback=cellbackground, colframe=cellborder]
\prompt{In}{incolor}{41}{\boxspacing}
\begin{Verbatim}[commandchars=\\\{\}]
\PY{n}{Kr}\PY{p}{(}\PY{p}{)}
\end{Verbatim}
\end{tcolorbox}
 
            
\prompt{Out}{outcolor}{41}{}
    
    $\displaystyle \frac{48 G^{2} M^{2}}{c^{4} r^{6}}$

    

    \hypertarget{soluuxe7uxf5es-de-sitter-schwarzschild}{%
\section{Soluções De
Sitter-Schwarzschild}\label{soluuxe7uxf5es-de-sitter-schwarzschild}}

    Com as soluções acima, podemos modificar um pouco nossas hipóteses e
estudar outras duas soluções muito importantes. A primeira é fundamental
no estudo da cosmologia: O espaço de Sitter, e outra solução fundamental
na física teórica atual: o espaço Anti-de Sitter. Em particular, vamos
derrubar as condições 3) e 4), ou seja, o espaço não é mais
assimptoticamente plano, e a constante cosmológica não é nula.

Toda a análise que fizemos na seção anterios, e na prática, a única
modificação será na equação de Einstein

\begin{equation}
R_{\mu\nu} - \Lambda g_{\mu\nu} = 0 
\end{equation}

portanto

    \begin{tcolorbox}[breakable, size=fbox, boxrule=1pt, pad at break*=1mm,colback=cellbackground, colframe=cellborder]
\prompt{In}{incolor}{15}{\boxspacing}
\begin{Verbatim}[commandchars=\\\{\}]
\PY{n}{Equation} \PY{o}{=} \PY{p}{[}\PY{p}{(}\PY{n}{Ric}\PY{p}{[}\PY{n}{i}\PY{p}{,}\PY{n}{j}\PY{p}{]} \PY{o}{\PYZhy{}} \PY{n}{Lb}\PY{o}{*}\PY{n}{g}\PY{p}{[}\PY{n}{i}\PY{p}{,}\PY{n}{j}\PY{p}{]}\PY{p}{)}\PY{o}{.}\PY{n}{simplify}\PY{p}{(}\PY{p}{)} \PY{k}{for} \PY{n}{i} \PY{o+ow}{in} \PY{n+nb}{range}\PY{p}{(}\PY{n}{rank}\PY{p}{)} \PY{k}{for} \PY{n}{j} \PY{o+ow}{in} \PY{n+nb}{range}\PY{p}{(}\PY{n}{rank}\PY{p}{)} \PY{k}{if} \PY{n}{Ric}\PY{p}{[}\PY{n}{i}\PY{p}{,}\PY{n}{j}\PY{p}{]} \PY{o}{!=} \PY{l+m+mi}{0}\PY{p}{]}
\end{Verbatim}
\end{tcolorbox}

    \begin{tcolorbox}[breakable, size=fbox, boxrule=1pt, pad at break*=1mm,colback=cellbackground, colframe=cellborder]
\prompt{In}{incolor}{16}{\boxspacing}
\begin{Verbatim}[commandchars=\\\{\}]
\PY{c+c1}{\PYZsh{} Para ver o resultado explícito, fazemos}
\PY{k}{for} \PY{n}{k} \PY{o+ow}{in} \PY{n+nb}{range}\PY{p}{(}\PY{n+nb}{len}\PY{p}{(}\PY{n}{Equation}\PY{p}{)}\PY{p}{)}\PY{p}{:}
    \PY{n+nb}{print}\PY{p}{(}\PY{l+s+sa}{f}\PY{l+s+s2}{\PYZdq{}}\PY{l+s+s2}{+++++++++ Equation(}\PY{l+s+si}{\PYZob{}}\PY{n}{k}\PY{l+s+si}{\PYZcb{}}\PY{l+s+s2}{,}\PY{l+s+si}{\PYZob{}}\PY{n}{k}\PY{l+s+si}{\PYZcb{}}\PY{l+s+s2}{) ++++++++++++++++}\PY{l+s+s2}{\PYZdq{}}\PY{p}{)}
    \PY{n}{display}\PY{p}{(}\PY{n}{Equation}\PY{p}{[}\PY{n}{k}\PY{p}{]}\PY{p}{)}
    \PY{n+nb}{print}\PY{p}{(}\PY{l+s+s2}{\PYZdq{}}\PY{l+s+s2}{ }\PY{l+s+s2}{\PYZdq{}}\PY{p}{)}
\end{Verbatim}
\end{tcolorbox}

    \begin{Verbatim}[commandchars=\\\{\}]
+++++++++ Equation(0,0) ++++++++++++++++
    \end{Verbatim}

    $\displaystyle \frac{Lb r e^{f_{1}{\left(r \right)}} + \frac{\left(r \left(\left(\frac{d}{d r} f_{1}{\left(r \right)}\right)^{2} - \frac{d}{d r} f_{1}{\left(r \right)} \frac{d}{d r} f_{2}{\left(r \right)} + 2 \frac{d^{2}}{d r^{2}} f_{1}{\left(r \right)}\right) + 4 \frac{d}{d r} f_{1}{\left(r \right)}\right) e^{f_{1}{\left(r \right)} - f_{2}{\left(r \right)}}}{4}}{r}$

    
    \begin{Verbatim}[commandchars=\\\{\}]

+++++++++ Equation(1,1) ++++++++++++++++
    \end{Verbatim}

    $\displaystyle - Lb e^{f_{2}{\left(r \right)}} - \frac{\left(\frac{d}{d r} f_{1}{\left(r \right)}\right)^{2}}{4} + \frac{\frac{d}{d r} f_{1}{\left(r \right)} \frac{d}{d r} f_{2}{\left(r \right)}}{4} - \frac{\frac{d^{2}}{d r^{2}} f_{1}{\left(r \right)}}{2} + \frac{\frac{d}{d r} f_{2}{\left(r \right)}}{r}$

    
    \begin{Verbatim}[commandchars=\\\{\}]

+++++++++ Equation(2,2) ++++++++++++++++
    \end{Verbatim}

    $\displaystyle \left(- Lb r^{2} e^{f_{2}{\left(r \right)}} - \frac{r \frac{d}{d r} f_{1}{\left(r \right)}}{2} + \frac{r \frac{d}{d r} f_{2}{\left(r \right)}}{2} + e^{f_{2}{\left(r \right)}} - 1\right) e^{- f_{2}{\left(r \right)}}$

    
    \begin{Verbatim}[commandchars=\\\{\}]

+++++++++ Equation(3,3) ++++++++++++++++
    \end{Verbatim}

    $\displaystyle \left(- Lb r^{2} e^{f_{2}{\left(r \right)}} - \frac{r \frac{d}{d r} f_{1}{\left(r \right)}}{2} + \frac{r \frac{d}{d r} f_{2}{\left(r \right)}}{2} + e^{f_{2}{\left(r \right)}} - 1\right) e^{- f_{2}{\left(r \right)}} \sin^{2}{\left(th \right)}$

    
    \begin{Verbatim}[commandchars=\\\{\}]

    \end{Verbatim}

    \(\clubsuit\) \textbf{1.} Novamente, apenas 3 equações são
independentes. É simples verificar que \begin{equation}
(R_{33} - \Lambda g_{33}) - sin^2\theta (R_{2,2} - \Lambda g_{22}) = 0
\end{equation}

    \begin{tcolorbox}[breakable, size=fbox, boxrule=1pt, pad at break*=1mm,colback=cellbackground, colframe=cellborder]
\prompt{In}{incolor}{17}{\boxspacing}
\begin{Verbatim}[commandchars=\\\{\}]
\PY{p}{(}\PY{n}{Equation}\PY{p}{[}\PY{l+m+mi}{3}\PY{p}{]} \PY{o}{\PYZhy{}} \PY{n}{Equation}\PY{p}{[}\PY{l+m+mi}{2}\PY{p}{]}\PY{o}{*}\PY{p}{(}\PY{n}{sin}\PY{p}{(}\PY{n}{th}\PY{p}{)}\PY{o}{*}\PY{o}{*}\PY{l+m+mi}{2}\PY{p}{)}\PY{p}{)}\PY{o}{.}\PY{n}{simplify}\PY{p}{(}\PY{p}{)}
\end{Verbatim}
\end{tcolorbox}
 
            
\prompt{Out}{outcolor}{17}{}
    
    $\displaystyle 0$

    

    \(\clubsuit\) \textbf{2.} Novamente, as 2 primeiras equações possuem uma
estrutura muito semelhante, então faremos a operação a seguir

\begin{equation}
r e^{-f_1+f_2} (R_{00} - \Lambda g_{00}) + r(R_{11} - \Lambda g_{11})  = 0
\end{equation}

Exatamente como fizemos como no estudo da solução Schwarzschild.

    \begin{tcolorbox}[breakable, size=fbox, boxrule=1pt, pad at break*=1mm,colback=cellbackground, colframe=cellborder]
\prompt{In}{incolor}{18}{\boxspacing}
\begin{Verbatim}[commandchars=\\\{\}]
\PY{n}{r}\PY{o}{*}\PY{p}{(}\PY{n}{exp}\PY{p}{(}\PY{o}{\PYZhy{}}\PY{n}{f1}\PY{o}{+}\PY{n}{f2}\PY{p}{)}\PY{o}{*}\PY{n}{Equation}\PY{p}{[}\PY{l+m+mi}{0}\PY{p}{]} \PY{o}{+} \PY{n}{Equation}\PY{p}{[}\PY{l+m+mi}{1}\PY{p}{]}\PY{p}{)}\PY{o}{.}\PY{n}{simplify}\PY{p}{(}\PY{p}{)}
\end{Verbatim}
\end{tcolorbox}
 
            
\prompt{Out}{outcolor}{18}{}
    
    $\displaystyle \frac{d}{d r} f_{1}{\left(r \right)} + \frac{d}{d r} f_{2}{\left(r \right)}$

    

    de onde concluímos que \begin{equation}  
f_1 + f_2 = c_0 \quad \Rightarrow \quad f_2 = -f_1  
\end{equation} onde \(c_0\) é uma constante que escolhemos como zero.

    \(\clubsuit\) \textbf{3.} Agora vamos usar essa condição na equação
\begin{equation}
R_{22} - \Lambda g_{22} = 0 
\end{equation}

    \begin{tcolorbox}[breakable, size=fbox, boxrule=1pt, pad at break*=1mm,colback=cellbackground, colframe=cellborder]
\prompt{In}{incolor}{21}{\boxspacing}
\begin{Verbatim}[commandchars=\\\{\}]
\PY{p}{(}\PY{n}{Equation}\PY{p}{[}\PY{l+m+mi}{2}\PY{p}{]}\PY{p}{)}\PY{o}{.}\PY{n}{simplify}\PY{p}{(}\PY{p}{)}
\end{Verbatim}
\end{tcolorbox}
 
            
\prompt{Out}{outcolor}{21}{}
    
    $\displaystyle \left(- Lb r^{2} e^{f_{2}{\left(r \right)}} - \frac{r \frac{d}{d r} f_{1}{\left(r \right)}}{2} + \frac{r \frac{d}{d r} f_{2}{\left(r \right)}}{2} + e^{f_{2}{\left(r \right)}} - 1\right) e^{- f_{2}{\left(r \right)}}$

    

    e a solução dessa expressão é dada por

\begin{equation}
e^{f_1} = 1 - \frac{2 G M}{c^2 r} - \frac{\Lambda r^2}{3}
\end{equation}

    E essa solução é conhecida como De Sitter-Schwarzschild e a métrica é
dada por

\begin{equation}
ds^2 = -\left(1 - \frac{2 G M}{c^2 r} - \frac{\Lambda r^2}{3}\right) c^2 dt^2 + \frac{dr^2}{\left(1 - \frac{2 G M}{c^2 r} - \frac{\Lambda r^2}{3}\right)} + r^2 d\theta^2 + r^2 \sin^2\theta d\phi^2 
\end{equation}

    O caso particular onde \(M=0\) é chamada de \emph{de Sitter} (dS) (se
\(\Lambda > 0\)) e \emph{anti-de Sitter} (se \(\Lambda < 0\)). Vamos
agora finalizar a análise calculando o escalar de curvatura e de
Kretschmann.

    \(\spadesuit\) \textbf{ESCALAR DE CURVATURA:} No caso da solução de
Schwarzschild, o escalar de curvatura é trivial pois a tensor de Ricci é
nulo. Nas soluções de Sitter, devemos analisar esse invariante.

    \begin{tcolorbox}[breakable, size=fbox, boxrule=1pt, pad at break*=1mm,colback=cellbackground, colframe=cellborder]
\prompt{In}{incolor}{14}{\boxspacing}
\begin{Verbatim}[commandchars=\\\{\}]
\PY{k}{def} \PY{n+nf}{RCurv}\PY{p}{(}\PY{p}{)}\PY{p}{:}
    \PY{n}{curv} \PY{o}{=} \PY{l+m+mi}{0}
    \PY{n}{n} \PY{o}{=} \PY{l+m+mi}{0}
    \PY{k}{for} \PY{n}{k} \PY{o+ow}{in} \PY{n+nb}{range}\PY{p}{(}\PY{n}{rank}\PY{p}{)}\PY{p}{:}
        \PY{n}{curv} \PY{o}{=} \PY{n}{curv} \PY{o}{+} \PY{n}{gI}\PY{p}{[}\PY{n}{k}\PY{p}{,}\PY{n}{k}\PY{p}{]}\PY{o}{*}\PY{n}{Ric}\PY{p}{[}\PY{n}{k}\PY{p}{,}\PY{n}{k}\PY{p}{]}
        \PY{n}{r} \PY{o}{=} \PY{n}{curv}\PY{o}{.}\PY{n}{simplify}\PY{p}{(}\PY{p}{)} 
    \PY{k}{return} \PY{n}{r}
\end{Verbatim}
\end{tcolorbox}

    \begin{tcolorbox}[breakable, size=fbox, boxrule=1pt, pad at break*=1mm,colback=cellbackground, colframe=cellborder]
\prompt{In}{incolor}{15}{\boxspacing}
\begin{Verbatim}[commandchars=\\\{\}]
\PY{n}{RCurv}\PY{p}{(}\PY{p}{)}
\end{Verbatim}
\end{tcolorbox}
 
            
\prompt{Out}{outcolor}{15}{}
    
    $\displaystyle 4 Lb$

    

    Ou seja, \(R = 4 \Lambda\) como esperado.

    \(\spadesuit\) \textbf{ESCALAR DE KRETSCHMANN:} Por completeza, podemos
calcular esse escalar como

    \begin{tcolorbox}[breakable, size=fbox, boxrule=1pt, pad at break*=1mm,colback=cellbackground, colframe=cellborder]
\prompt{In}{incolor}{18}{\boxspacing}
\begin{Verbatim}[commandchars=\\\{\}]
\PY{n}{Kr}\PY{p}{(}\PY{p}{)}\PY{o}{.}\PY{n}{simplify}\PY{p}{(}\PY{p}{)}
\end{Verbatim}
\end{tcolorbox}
 
            
\prompt{Out}{outcolor}{18}{}
    
    $\displaystyle \frac{48 G^{2} M^{2}}{c^{4} r^{6}} + \frac{8 Lb^{2}}{3}$

    

    Veja que ainda continuamos com a singularidade no ponto \(r=0\), mas
nenhum outro ponto é mal definido. Em particular, observe que o caso
onde \(M=0\) e \(\Lambda > 0\), temos que a métrica diverge no ponto
\(r = \sqrt{3 / \lambda}\). Mais uma vez, esse ponto é um artefato da
métrica. Nesse caso, chamamos a superfície definida por esses pontos de
horizonte cosmológico.


    % Add a bibliography block to the postdoc
    
    
    
\end{document}
